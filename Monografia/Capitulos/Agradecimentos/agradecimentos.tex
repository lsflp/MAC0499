\chapter*{Agradecimentos e parte subjetiva}

\lettrine{A}{ntes} de tudo, gostaria de retribuir àqueles que tornaram a existência desse trabalho possível. Primeiramente, agradeço a Alan Turing, considerado o pai da Ciência da Computação e a Ada Lovelace, conhecida como a primeira programadora da história.

Sou grato também a todos os professores do Instituto de Matemática e Estatística, em especial os do Departamento Ciência da Computação com os quais tive aula, alguns citados aqui: A. Fujita; A. Mandel; A. Melo; C. Ferreira; D. Batista; E. Birgin; J. Barrera; J. de Pina Jr.; J. Ferreira; M. Gubitoso; P. Feofiloff; W. Mascarenhas e Y. Kohayakawa. 

Gostaria de dar destaque para aqueles que ajudaram na minha ênfase em Inteligência Artificial: F. Kon; L. Barros; N. Hirata e R. Wassermann, a orientadora deste trabalho.

Reconheço também a ajuda dos alunos de pós-graduação em Computação do IME-USP: F. Resina e V. Matos, que me deram grande suporte durante a execução deste. 

Agradeço muito à minha família. Sem a assistência que eles me forneceram durante toda a minha vida, eu não teria chegado até aqui.

Este trabalho não é parecido com nada que eu já fiz. Foi a primeira vez que eu fiz um estudo teórico mais profundo sobre uma área tão específica. As leituras começaram com assuntos que eu já tinha noções, e foram evoluindo para matérias cujo meu conhecimento era quase nulo. A parte teórica me surpreendeu, pois foi a que tive mais prazer em realizar.

A parte prática também me deixou pasmo, mas por outro motivo. A princípio, achei que teria problemas com a parte teórica, mas aconteceu o contrário. Mesmo trabalhando com ferramentas já conhecidas, tive dificuldade em conseguir fazer o projeto.

Em suma, acredito que saio outra pessoa depois da entrega deste trabalho. Aprendi muito durante sua execução, não apenas sobre o seu tema, mas tudo relacionado, como pesquisa, leitura, escrita e desenvolvimento. Além disso, existe a experiência relacionada ao meu ritmo de trabalho, que com certeza irei me lembrar para sempre.