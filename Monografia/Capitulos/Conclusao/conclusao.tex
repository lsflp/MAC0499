\chapter{Conclusão}

Este Trabalho de Formatura Supervisionado consistiu num estudo sobre Revisão de Crenças e em assuntos que servem de base para o seu entendimento: ontologias e Lógicas de Descrição. Depois disso, foi construído um \textit{plug-in} que implementa operações de Revisão de Crenças, com o auxílio de ferramentas como o \textit{Protégé}.

A parte teórica foi a que levou mais tempo de estudo. Mais da metade do tempo relacionado a esta disciplina foi dedicada a esse componente do trabalho.

Implementar o \textit{plug-in} em si não foi difícil. No entanto, foi necessário um tempo de adaptação com as construções anteriores e com as ferramentas utilizadas (em especial a \textit{OWL API}). A adaptação levou mais tempo do que a programação.

Esse trabalho deixa várias oportunidades de continuidade abertas, tais como: estudo e implementação da Revisão \textit{Partial Meet} \citep{revisaoRibeiro2} no \textit{plug-in}; uma adaptação para que o \textit{plug-in} aqui construído funcione dentro do \textit{Protégé}, sem o auxílio de uma janela de terminal; comparações mais complexas entre os resultados das operações com fórmulas relacionadas, uma vez que as feitas aqui são bem rudimentares; testes usando diferentes funções de seleção e incisão; experiências com arquivos mais robustos, assim como os trabalhados por Cóbe \citep{revisaoCobe}, que utilizou 4 conjuntos de teste diferentes.