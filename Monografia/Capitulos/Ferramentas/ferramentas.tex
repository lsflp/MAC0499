\chapter{Ferramentas Computacionais}

\lettrine{O}{} interesse no estudo de Desenvolvimento de Ontologias, Lógicas de Descrição e Revisão de Crenças, deve-se, em parte, às aplicações computacionais que tais áreas possuem. Neste capítulo, algumas delas serão abordadas.

\section{OWL}

O final da década de 1990 foi a época de surgimento da ideia de Web Semântica. Essa ideia, já consolidada hoje, representa uma extensão da \textit{World Wide Web} onde é possível o trabalho cooperativo entre as máquinas e os seres humanos. Ele ocorre a partir dos significados que são dados aos conteúdos disponíveis na rede para que ele seja compreensível por ambos os agentes \cite{ferramentasHerman}.

No início dos anos 2000, a \href{https://www.w3.org}{W3C} (sigla para Consórcio para a \textit{World Wide Web}) fundou o "\textit{Web Ontology Working Group}" \cite{ferramentasGrupo}, para que alguma Linguagem de Representação fosse criada. 

Em julho de 2002, os primeiros rascunhos foram publicados, e em fevereiro de 2004 \cite{ferramentasReco}, a OWL (\textit{Web Ontology Language}) tornou-se um padrão recomendado pela W3C para o processamento de ontologias.

Em outubro de 2007 \cite{ferramentasOWLGrupo2}, um novo grupo foi concebido para adicionar alguns recursos à linguagem. Dois anos depois, a W3C fez o lançamento da OWL 2, que seria compatível com editores e raciocinadores semânticos. Ela foi recomendada oficialmente pela W3C em dezembro de 2012 \cite{ferramentasOWLReco2}.

Abaixo temos algumas características das duas versões da OWL, que foi baseada em RDF e RDFS.

\subsection{OWL Web Ontology Language}

Possui três sublinguagens \cite{ferramentasOWL1}, cada uma com suas particularidades:

\begin{itemize}
	\item \textit{Lite}: Sublinguagem mais simples, feita para hierarquias de classificação com restrições simples. Suporta restrições de cardinalidade simples (apenas 0 ou 1).  Foi demonstrado que OWL-Lite é equivalente a $ \mathcal{SHIF^{(D)}} $.
	\item DL: É equivalente a $ \mathcal{SHOIN^{(D)}} $. Oferece expressividade máxima evitando alguns problemas de decidibilidade. Inclui todos os construtores da OWL. Possui mais complexidade formal do que a sublinguagem \textit{Lite}.
	\item \textit{Full}: Versão mais completa, oferecendo expressividade máxima e a liberdade sintática do RDF, só que sem garantias computacionais. Nessa sublinguagem, pode acontecer de uma Classe ser tratada como uma coleção de indivíduos ou como um indivíduo, o que acarreta problemas de decidibilidade. Não corresponde a uma lógica de descrição.
\end{itemize}

\subsection{OWL 2 Web Ontology Language}

Assim como a primeira versão, também possui sublinguagens \cite{ferramentasOWL2}. Elas são:

\begin{itemize}
	\item EL: Corresponde à lógica $ \mathcal{EL} $++. Funciona bem para ontologias com muitas classes e/ou propriedades. Permite a existência de algoritmos de inferência polinimiais.
	\item QL: Baseada na lógica de descrição DL-\textit{Lite}, foi feita para aplicações em que o volume de instâncias é muito grande e que a consulta de dados é a tarefa mais realizada. Tais consultas podem ser implementadas usando sistemas de bancos de dados convencionais.
	\item RL: Criada para ontologias que precisam de raciocínio escalável, ou seja, que são possivelmente grandes em número de instâncias, mas que precisam de uma sublinguagem que mantenha um bom poder expressivo. É baseada em uma lógica de descrição chamada DLP.
	\item DL: Assim como no lançamento anterior, oferece bastante expressividade, sem esbarrar em problemas de execução por causa da indecibilidade. Ela pode ser mapeada para a LD $ \mathcal{SROIQ^{(D)}} $. 
	\item \textit{Full}: Análoga à da primeira versão. Muito expressiva, porém, indecidível.
\end{itemize}

{\color{red} Colocar parte do arquivo owl da ontologia?}

\section{\textit{Protégé}}

O \href{https://protege.stanford.edu}{\textit{Protégé}} é um editor semântico, compatível com a OWL 2. Ele foi desenvolvido pelo Centro de Pesquisa para Informática Biomédica da Universidade de Stanford, em colaboração com alguns programadores da Universidade de Manchester \cite{ferramentasProtege}. Sua primeira versão foi lançada em 1999, e a atual é de 2017.

Seguindo os princípios do \textit{Software} Livre, ele é um arcabouço gratuito e de código aberto, feito para a construção de sistemas inteligentes, com o uso ou não de uma interface gráfica.

Assim como o ambiente de desenvolvimento integrado Eclipse, o \textit{Protégé} é bastante flexível porque é possível desenvolver uma grande variedade \textit{plug-ins} para serem acoplados a ele.

Para os exemplos dessa seção, será usado como exemplo parte da ontologia criada neste estudo:

\begin{itemize}
	\item Os conceitos: 
	\begin{itemize}
		\item \texttt{Musica} $ \equiv $ \texttt{Cancao} $ \sqcup $ \texttt{HinoNacional}.
		\item \texttt{Pop} $ \sqsubseteq $ \texttt{Cancao};
		\item \texttt{Pessoa} $ \equiv $ \texttt{Cantor} $ \sqcup $ \texttt{Compositor}.
	\end{itemize}
	\item As propriedades:
	\begin{itemize}
		\item \textbf{\texttt{canta}}, com domínio \texttt{Cantor} e contradomínio \texttt{Musica};
		\item \textbf{\texttt{cantadaPor}}, inversa da propriedade acima.
	\end{itemize}
	\item As instâncias e asserções:
	\begin{itemize}
		\item \texttt{MarinaAndTheDiamonds}, instância da classe \texttt{Cantor};
		\item \texttt{Primadonna}, instância da classe \texttt{Pop};
		\item \texttt{MarinaAndTheDiamonds \textbf{canta} Primadonna}.
		\item \texttt{Froot}, instância da classe \texttt{Pop};
		\item \texttt{MarinaAndTheDiamonds \textbf{canta} Froot}.	
	\end{itemize}
\end{itemize}

\subsection{Raciocinadores}

Um aspecto interessante sobre esse arcabouço é o uso de plug-ins de \textit{raciocínio}, como o {\color{red} Colocar o que utilizei.}. A partir deles, é possível fazer inferências. 

{\color{red} Colocar o exemplo da inferência que Froot é cantada por Marina and the Diamonds.}

\subsection{Buscas}

A partir de uma ontologia e de uma base de dados, é possível fazer buscas no arcabouço, utilizando, entre várias linguagens de busca, o SPARQL, por exemplo.

SPARQL é um acrônimo recursivo para \textit{SPARQL Protocol and RDF Query Language} \cite{ferramentasSPARQL}. Tem uma sintaxe que lembra a do SQL, e com seu uso, é possível tratar a ontologia como uma base de dados.  

Vale notar que a busca funciona apenas com a terminologia e as asserções gravadas no arquivo da ontologia. Para que as buscas funcionem sobre as inferências, é necessário exportá-las para um novo arquivo.

{\color{red} Colocar o exemplo de quais são as músicas pop que a Marina canta.}

{\color{red} Colocar o logo do Protégé.}