\chapter{Implementação do \textit{plug-in}}
\label{chap:implementacao}

\lettrine{O}{} \textit{plug-in} construído para este trabalho é, na verdade, uma reunião de implementações realizadas em trabalhos anteriores. As operações cobertas pelo \textit{plug-in} e suas respectivas construções anteriores são:

\begin{description}
	\item[Contração] O \textit{plug-in} possui os dois construtores descritos neste trabalho, as Contrações \textit{Partial Meet} e a \textit{Kernel}. Ambas foram baseadas em implementações de algoritmos feitas na tese de doutorado de Cóbe  \citep{revisaoCobe}.
	\item[Revisão \textit{Kernel}] Para essa operação, os códigos feitos por Resina para o seu trabalho de mestrado \citep{logicaResina} foram reconstruídos, com o auxílio dos estudos de Ribeiro \citep{revisaoRibeiro2}. 
	\item[Pseudocontração SRW] Essa operação foi completamente absorvida ao \textit{plug-in} do código feito por Matos, em seu trabalho de conclusão de curso. \citep{logicaMatos}
\end{description} 

Todo o código-fonte está disponível no \href{https://github.com/lsflp/ontology-repair/}{GitHub}, assim como os arquivos
compilados e instruções de compilação e execução.

O programa consiste em uma interface com o usuário pelo terminal. Ele recebe os parâmetros pela linha de comando. A saída é um arquivo OWL com uma ontologia que pode ser aberto no \textit{Protégé}. As informações de entrada e saída podem ser acessadas no \href{https://github.com/lsflp/ontology-repair/blob/master/README.md}{README} do projeto.

\section{Desenvolvimento}

O projeto foi inteiramente feito no IntelliJ versão 2018.2 \footnote{\url{https://www.jetbrains.com/idea/}}. Esse \textit{software} não é gratuito, mas oferece uma versão de uso para estudantes. As dependências foram gerenciadas pelo \textit{Apache Maven} 3.5.2 \footnote{\url{https://maven.apache.org/}}.

Para as lógicas internas, são necessários a \textit{OWL API} 5.1.6 \footnote{\url{http://owlcs.github.io/owlapi/}}, usada para todo o trabalho com os axiomas e o \textit{HermiT Reasoner} 1.3.8 \footnote{\url{http://www.hermit-reasoner.com/}}, um motor de inferências. Para facilitar a entrada dos parâmetros, foi utilizado o \textit{JCommander} 1.58 \footnote{\url{http://jcommander.org}}.

\section{Algoritmo \textit{BlackBox}}

A implementação das quatro operações tem a sua parte principal em um algoritmo \textit{BlackBox}, baseados nos estudos de Resina, que provou sua correção \citep{logicaResina}. Ele é chamado assim pois não precisa de um motor de inferência. É necessário apenas decidir se uma base de conhecimento implica certo axioma, ou se a base é consistente. Ele possui variações para o cálculo do conjunto-resíduo ou para o cálculo do conjunto-\textit{kernel};

Para efeitos de otimização, foi colocado um limite tanto nas filas utilizadas em alguns dos algoritmos abaixo quanto nos conjuntos que serão retornados. Tais limites podem ser definidos pelo usuário, opcionalmente.

\subsection{\textit{BlackBox} para o conjunto-\textit{kernel} para a Contração}

Para o cálculo do conjunto-\textit{kernel}, usado na Contração \textit{Kernel}, foi utilizada uma adaptação do código de Cóbe \citep{revisaoCobe}. Ela consiste em duas funções:

\begin{description}
	\item[\textsc{KernelBlackBox}] É uma função que recebe um conjunto de axiomas $ B $ e uma sentença $ A $. Ela constrói um conjunto minimal de $ B $ que implica $ A $. Essa função é uma adaptação da estudada por Resina \citep{logicaResina}, e se divide em duas partes: expansão, onde são adicionados todos os axiomas de B, caso $ B \vdash A $, a um conjunto $ B' $, definido previamente como vazio; e encolhimento, onde cada elemento $ \beta $ de $ B' $ é analisado individualmente. Caso $ B' \setminus \{\beta\}$ implique $ A $, $ \beta $ é apagado de $ B' $. Logo, o conjunto devolvido é um conjunto minimal de $ B $ que implica $ A $, portanto, um elemento do Kernel. Seu código é o seguinte: \\
	\begin{algorithmic}
		\Function{KernelBlackBox}{B, A}
		\State $ B' \gets \varnothing $
		\If{$ B \vdash A $}
		\State $ B' \gets B $
		\EndIf
		\ForAll{$ \beta \in B' $}
		\If{$ B' \setminus \{\beta\} \vdash A $}
		\State $ B' \gets B' \cup \{\beta\}$
		\EndIf
		\EndFor
		\EndFunction
	\end{algorithmic}

	\item[\textsc{KernelSet}] É uma função que recebe os mesmos parâmetros citados acima, e utiliza a função supracitada. Ela começa com um elemento do Kernel. Assim como o construtor do conjunto-resíduo, ele constrói uma árvore, percorrendo em largura. Para cada elemento $ Hn $ da fila, o algoritmo o remove de $ B $ e, se $ B $ ainda implica $ A $, computa-se o menor subconjunto S de $ B \setminus Hn $ que implica $ A $, e então, tem-se em S um novo elemento do Kernel. Depois disso, $ B $ é restaurado. Seu código está abaixo: \\
	
	\begin{algorithmic}
		\Function{KernelSet}{B, A}
		\If{$ B \nvdash A $}
		\State \Return $ \varnothing $
		\EndIf
		\State $ fila \gets $ fila vazia
		\State $ S \gets $ \Call{KernelBlackBox}{B, A}
		\State $ kernel \gets \{S\} $
		\ForAll{$ s \in S $}
		\State coloque $ s $ em $ fila $
		\EndFor
		\While{$ fila $ não está vazia}
		\State {$ Hn \gets $ o próximo de $ fila $}
		\State $ B \gets B \setminus Hn $
		\If{$ B \vdash A $}
		\State $ S  \gets $ \Call{KernelBlackBox}{B, A}
		\State $ kernel \gets kernel \cup \{S\} $
		\ForAll{$ s \in S $}
		\State coloque $ Hn \cup \{s\} $ em $ fila $
		\EndFor
		\EndIf
		\State $ B \gets B \cup Hn $
		\EndWhile
		\State \Return $ kernel $
		\EndFunction
	\end{algorithmic}
\end{description}

\subsection{\textit{BlackBox} para o conjunto-resíduo}

Para o cálculo do conjunto-resíduo, utilizado na Contração \textit{Partial Meet} e na Pseudocontração SRW, foi utilizada a implementação adaptada por Matos da implementação original de Cóbe \citep{logicaMatos}. Ela consiste em duas funções:

\begin{description}
	\item[\textsc{RemainderBlackBox}] É uma função que recebe um conjunto de axiomas $ B $, uma sentença $ A $ e um conjunto de axiomas $ X $, tal que $ X \nvdash A $, e constrói um elemento do conjunto resíduo $ (B \bot A) $, que contém todos os elementos de $ X $. O conjunto devolvido $ X' $ é tal que $ X' \subseteq X \in B \bot A $. O método começa com $ X $  e acrescenta todos os axiomas de $ B $ que não façam o conjunto resultante implicar $ A $. De acordo como o laço principal é implementado, resultados diferentes podem ser alcançados. No entanto, isso não é um problema, porque a função que chama esta só pede um item do conjunto-resíduo. O seu código segue abaixo: \\
	\begin{algorithmic}
		\Function{RemainderBlackBox}{B, A, X}
		\State $ X' \gets X$
		\ForAll{$ \beta \in B \setminus X $}
		\If{$ X' \cup \{\beta\} \nvdash A $}
		\State $ X' \gets X' \cup \{\beta\} $
		\EndIf 
		\EndFor
		\State \Return $ X' $
		\EndFunction
	\end{algorithmic}
	
	\item[\textsc{RemainderSet}] Usando a função acima, ela constrói o conjunto-resíduo. Seja $ X $ um conjunto, tal que $ X \nvdash A$, inicialmente vazio. Implicitamente, uma árvore é construída. Sua raiz é um elemento do conjunto-resíduo obtido a partir de $ X $, e para cada axioma $ s $ fora desse conjunto, se $ X \cup \{s\} \nvdash A $, o algoritmo cria um nó filho na árvore com um conjunto-resíduo obtido a partir de $ X' = X \cup \{s\} $. Como se deseja apenas gerar o conjunto, a árvore não é construída por completo. Ao invés disso, os elementos são criados como se a árvore fosse percorrida por uma busca em largura, por isso o uso da fila. O seu código segue abaixo: \\
	\begin{algorithmic}
		\Function{RemainderSet}{B, A}
		\State $ fila \gets $ fila vazia
		\State $ S \gets $ \Call{RemainderBlackBox}{B, A, $ \varnothing $}
		\State $ remainder \gets \{S\} $
		\ForAll{$ s \in B \setminus S $}
		\State coloque $ s $ em $ fila $
		\EndFor
		\While{$ fila $ não está vazia}
		\State {$ Hn \gets $ o próximo de $ fila $}
		\If{$ Hn \nvdash A $}
		\State $ S \gets $ \Call{RemainderBlackBox}{B, A, Hn}
		\State $ remainder \gets remainder \cup \{S\} $
		\ForAll{$ s \in B \setminus S $}
		\State coloque $ Hn \cup \{s\} $ em $ fila $
		\EndFor
		\EndIf
		\EndWhile
		\State \Return $ remainder $
		\EndFunction
	\end{algorithmic}
\end{description}

\subsection{\textit{BlackBox} para o conjunto-\textit{kernel} para a Revisão}

O cálculo do conjunto-\textit{kernel} da Revisão \textit{Kernel} é feito a partir de uma construção feita em 2008 \citep{revisaoRibeiro2}. Assim como os outros, consiste em duas funções:

\begin{description}
	\item[\textsc{RevisionKernelBlackBox}] É uma função que recebe o resultado de uma outra operação: a Expansão. Não é importante, no início, se o conjunto de entrada é consistente. Seu funcionamento é semelhante com os \textsc{BlackBox} anteriores. Possui uma parte de expansão e outra de encolhimento. A única diferença aqui, é que em vez de verificar se $ B' \vdash \alpha $ ($ B' $ um conjunto inicialmente vazio), checa-se se $ B' $ é consistente/coerente. A função devolve um conjunto minimal inconsistente. O algoritmo segue abaixo: \\
	\begin{algorithmic}
		\Function{RevisionKernelBlackBox}{$ B + \alpha $}
		\State $ B' \gets \varnothing $
		\ForAll{$ \beta \in B + \alpha $}
		\State $ B' \gets B' \cup \{\beta\} $
		\If{$ B' $ é inconsistente/incoerente}
		\State Pare
		\EndIf
		\EndFor
		\ForAll{$ \epsilon \in B' $}
		\If{$ B' \setminus \{\epsilon\} $ é inconsistente/incoerente}
		\State $ B' \gets B' \setminus \{\epsilon\}  $
		\EndIf
		\EndFor
		\State \Return $ B' $
		\EndFunction
	\end{algorithmic}
	\item[\textsc{RevisionKernelSet}] Essa função também recebe o resultado da Expansão $ B + \alpha $. Diferentemente do conjunto resíduo, aqui a árvore é construída em profundidade. Como todos os algoritmos recursivos, possui um caso base e um geral. O caso base é quando o conjunto de entrada é consistente, e não há nada a ser feito. Para o caso geral, ela chama a função definida acima, e para todos os elementos do conjunto definido abaixo como $ S $, ele remove um dos elementos e tenta achar o conjunto de subconjuntos minimais inconsistentes para essa nova base. Seu código está abaixo: \\
	\begin{algorithmic}
		\Function{RevisionKernelSet}{$ B + \alpha $}
		\If{$ B + \alpha $ é consistente/coerente}
		\State \Return $ \varnothing $
		\EndIf
		\State $ S \gets $ \Call{RevisionKernelBlackBox}{$ B + \alpha $}
		\State $ B' \gets B' \cup \{S\} $
		\ForAll{$ s \in S $}
		\State $ B' \gets B' \cup $ \Call{RevisionKernelSet}{$ B + \alpha \setminus \{\beta\} $}
		\EndFor
		\State \Return $ B' $
		\EndFunction
	\end{algorithmic}
\end{description}

\section{Funções de seleção e incisão}

Duas das operações precisam de uma função de seleção $ \gamma $. A implementação é flexível, a partir de uma interface. A função que está codificada aqui, foi trazida de um trabalho anterior \citep{logicaMatos}, e devolve apenas um elemento do conjunto-resíduo.

As outras duas operações necessitam de uma função de incisão $ \sigma $. A implementação foi feita da mesma maneira, com o uso de uma interface. Ela devolve a união dos elementos do conjunto-\textit{kernel}.

\section{Exemplos da execução}

São feitos aqui alguns exemplos simples para ilustrar a funcionalidade do \textit{plug-in}. Todos os exemplos estão relacionados à ontologia discutida neste trabalho.

\subsection{Contração \textit{Kernel}}

Para a Contração \textit{Kernel}, será usado um exemplo análogo. As classes estudadas aqui são \texttt{Cancao}, \texttt{Pop} e \texttt{SynthPop}. Os axiomas da ontologia são os listados abaixo, e sua representação está na \autoref{img:ck1}.

\begin{itemize}
	\item \texttt{SynthPop} $ \sqsubseteq $ \texttt{Pop}
	\item \texttt{Pop} $ \sqsubseteq $ \texttt{Cancao}
\end{itemize}

Suponha que o subgênero \texttt{SynthPop} evoluiu e está distante do \texttt{Pop}, mas ainda mantém o nome. Portanto, não se deseja mais que ele seja uma subclasse de \texttt{Pop}.

\begin{figure}[H]
	\centering
	\includegraphics[width=0.5\textwidth]{Capitulos/Implementacao/ck1.png}
	\caption{As classes da ontologia de gêneros musicais.}
	\label{img:ck1}
\end{figure}

Com o comando abaixo, a operação é realizada. O resultado está na \autoref{img:ck2}.

\begin{small}
	\texttt{java -jar ontologyrepair-1.0.0-SNAPSHOT.jar -c --core-retainment -i cancao.owl \\ -o output.owl -f "SynthPop SubClassOf Pop"}
\end{small}

\begin{figure}[H]
	\centering
	\includegraphics[width=0.5\textwidth]{Capitulos/Implementacao/ck2.png}
	\caption{O resultado da operação Contração \textit{Kernel}.}
	\label{img:ck2}
\end{figure}


\subsection{Contração \textit{Partial Meet}}

Para a Contração \textit{Partial Meet}, o exemplo utilizado tem as classes \texttt{Musica}, \texttt{Cancao} e \texttt{HinoNacional}, e os seguintes axiomas: 

\begin{itemize}
	\item \texttt{HinoNacional} $ \sqsubseteq $ \texttt{Cancao}
	\item \texttt{Cancao} $ \sqsubseteq $ \texttt{Musica}
\end{itemize}

Sua representação fica como na \autoref{img:cpm1}.

\begin{figure}[H]
	\centering
	\includegraphics[width=0.5\textwidth]{Capitulos/Implementacao/cpm1.png}
	\caption{A estrutura da ontologia de tipos de música.}
	\label{img:cpm1}
\end{figure}

Como definido anteriormente, na verdade, \texttt{Cancao} e \texttt{HinoNacional} são classes na mesma hierarquia, portanto, precisamos remover \texttt{HinoNacional} $ \sqsubseteq $ \texttt{Cancao}. A operação é realizada com o seguinte comando:

\begin{small}
	\texttt{java -jar ontologyrepair-1.0.0-SNAPSHOT.jar -c --relevance -i musica.owl \\ -o output.owl -f "HinoNacional SubClassOf Cancao"}
\end{small}

Após a sua execução, acontece o desejado, mas aparentemente, esse não é o melhor resultado, como pode ser visto na \autoref{img:cpm2}.

\begin{figure}[H]
	\centering
	\includegraphics[width=0.5\textwidth]{Capitulos/Implementacao/cpm2.png}
	\caption{O resultado da operação Contração \textit{Partial Meet}.}
	\label{img:cpm2}
\end{figure}

\subsection{Revisão \textit{Kernel}}

O exemplo usado para a Revisão \textit{Kernel} será simples também. Sejam um fragmento da ontologia de música os axiomas abaixo e representação como na \autoref{img:r1}.

\begin{itemize}
	\item \texttt{FunkAmericano} $ \sqsubseteq $ \texttt{Cancao}
	\item \texttt{FunkBrasileiro} $ \sqsubseteq $ \texttt{FunkAmericano}
\end{itemize}

\begin{figure}[H]
	\centering
	\includegraphics[width=0.5\textwidth]{Capitulos/Implementacao/r1.png}
	\caption{A estrutura da outra ontologia de gêneros musicais.}
	\label{img:r1}
\end{figure}

No entanto, hoje já se classifica o Funk Brasileiro como distinto do Funk Americano. Logo, vamos adicionar à nossa base que os dois subgêneros são disjuntos.

Com o comando abaixo, acontece a operação:

\begin{small}
	\texttt{java -jar ontologyrepair-1.0.0-SNAPSHOT.jar -r -i funk.owl -o output.owl \\ -f "FunkBrasileiro DisjointWith FunkAmericano"}
\end{small}

O resultado é visto na \autoref{img:r2}. Note que a propriedade de disjunção está no produto final.

\begin{figure}[H]
	\centering
	\includegraphics[width=0.45\textwidth]{Capitulos/Implementacao/r2.png}
	\includegraphics[width=0.45\textwidth]{Capitulos/Implementacao/r3.png}
	\caption{O resultado da operação Revisão \textit{Kernel} e a presença da disjunção, na classe \texttt{FunkAmericano}, respectivamente.}
	\label{img:r2}
\end{figure}

\subsection{Pseudocontração SRW}
\label{sect:srw}

Para a Pseudocontração SRW, serão consideradas as classes \texttt{Cantor} e \texttt{Compositor} da ontologia musical, e o seguinte axioma: \texttt{Compositor} $ \sqsubseteq $ \texttt{Cantor}, com a representação exibida na \autoref{img:srw1}.

\begin{figure}[H]
	\centering
	\includegraphics[width=0.5\textwidth]{Capitulos/Implementacao/srw1.png}
	\caption{A estrutura da ontologia sobre ocupações na área de música.}
	\label{img:srw1}
\end{figure}

Ela possui um indivíduo, \texttt{markRonson}, pertencente à classe \texttt{Compositor}, como se mostra na \autoref{img:srw2}.

\begin{figure}[H]
	\centering
	\includegraphics[width=0.5\textwidth]{Capitulos/Implementacao/srw2.png}
	\caption{A instância da ontologia.}
	\label{img:srw2}
\end{figure}

Tudo isso implica que \texttt{markRonson} também é um \texttt{Cantor}, o que está errado. 

A operação pode ser feita com o seguinte comando:

\begin{small}
	\texttt{java -jar ontologyrepair-1.0.0-SNAPSHOT.jar -srw -i pessoa.owl -o output.owl \\ -f "markRonson Type: Cantor"}
\end{small}

Depois da operação, temos que \texttt{Compositor} deixa de ser subclasse de \texttt{Cantor}, e então, \texttt{markRonson} não pertence mais à classe \texttt{Cantor}, como se observa na \autoref{img:srw3}.

\begin{figure}[H]
	\centering
	\includegraphics[width=0.5\textwidth]{Capitulos/Implementacao/srw3.png}
	\caption{O resultado da operação Pseudocontração SRW.}
	\label{img:srw3}
\end{figure}
