\chapter{Introdução}

\section{Motivação}

A obtenção de novos conhecimentos é inerente a todo agente inteligente. Nós, seres humanos, sempre estamos em busca de aprender mais. Nesse processo, alguns velhos conhecimentos podem ser desconsiderados. Além disso, pode acontecer de uma nova informação entrar em conflito com o que já se conhece, causando uma reflexão.

O mesmo pode ser aplicado para os sistemas baseados em conhecimento. Esse ó o objetivo da área de Revisão de Crenças que possui, como seu marco inicial, o artigo \textit{On the Logic of Theory Change: Partial Meet Contraction and Revision Functions} \citep{revisaoAGM}. Ele apresenta três operações básicas: a Expansão, equivalente a receber um novo conhecimento; a Contração, análoga à apagar uma informação; e a Revisão, correspondente à adição de um novo conhecimento e com o cuidado de tirar tudo aquilo que pode entrar em conflito.

Para que os sistemas computacionais possam passar por uma revisão de suas crenças, é necessário que haja alguma codificação do conhecimento e do sistema. Para tal são usadas as ontologias, que são uma reunião de axiomas que modelam certa parte de um domínio do conhecimento. As ontologias são compostas por sentenças em Lógicas de Descrição, por conta do formalismo que essas últimas possuem.

Existe um padrão para a representação de Ontologias. Ele é conhecido por OWL (\textit{Web Ontology Language}) \citep{ferramentasOWL2}. Possui diversas linguagens, que acompanham sublinguagens equivalentes de Lógicas de Descrição. Para tornar a construção e uso de ontologias mais fácil, foi desenvolvido o \textit{Protégé} \footnote{\url{http://protege.stanford.edu/}}, pelo \textit{Stanford Center for Biomedical Informatics Research}, da Universidade de Stanford. A integração com diversos \textit{plug-ins} o tornou muito popular.  

Graças a diversos estudos, foi construído um \textit{plug-in} que possui quatro operações: a Contração \textit{Kernel} \citep{revisaoHansson5}, a Contração \textit{Partial Meet} \citep{revisaoAGM}, a Pseu\-do\-con\-tra\-ção SRW \citep{revisaoSantos} e a Revisão \textit{Kernel} \citep{revisaoRibeiro2}. Foram feitos testes para ver o comportamento desses \textit{plug-ins}, frente a diferentes ontologias.

\section{Estrutura do trabalho}

O capítulo \ref{chap:ontologias} mostra estudos na área de Ontologias, passando por definições teóricas, seus componentes, sua montagem e usabilidade. No capítulo \ref{chap:logicas} é feito um estudo sobre as Lógicas de Descrição, fazendo um rápido paralelo com as ontologias, comentando sobre possíveis linguagens e escrevendo sobre interpretação, consequência lógica e consistênicia. O capítulo \ref{chap:revisao} apresenta as operações de Revisão de Crenças desde sua motivação, passa pelas operações clássicas e termina com alguns construtores. Esses três capítulos compõem a parte teórica desse trabalho.

A parte prática é composta por três capítulos. O capítulo \ref{chap:ferramentas} exibe um rápido panorama de como as Ontologias são representadas computacionalmente, focando-se na linguagem OWL e no programa \textit{Protégé}. No capítulo \ref{chap:implementacao} é descrita a implementação do \textit{plug-in} de Revisão de Crenças prometido para este trabalho. Por fim, o capítulo \ref{chap:testes} mostra testes de performance e de comparação do \textit{plug-in} implementado.

No final do trabalho, há uma seção de agradecimentos e de observações quanto ao trabalho.