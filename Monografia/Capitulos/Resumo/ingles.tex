\chapter*{Abstract}

\lettrine{O}{ntologies} are systems used to store knowledge from some domain, such as health or cinema. They are based in classes, their properties and the relationships among them. Description Logics, subsets of First Order Logics, can be used to represent ontologies.

The inclusion of new information may conflict with parts of the previous knowledge. The repair of the ontology becomes something necessary, in order to restore its consistency. Belief Revision techniques are used to achieve that.

A system used to build and use ontologies is called Protégé. One can define classes, axioms and relationships of a ontology with it. With the help of some plug-ins, some inferences can be made, and even inconsistencies can be found.

In this work, a theoretical study is made, starting with the ontologies' origin, passing through Description Logics and ending with Belief Revision. Furthermore, using an already built plug-in and several other implementations, a program that executes some Belief Revision operations has been made.

\noindent \textbf{Keywords:} Belief Revision, OWL, Description Logics, Ontologies, Knowlege Representation.