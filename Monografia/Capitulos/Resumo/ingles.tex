\chapter*{Abstract}

\lettrine{O}{ntologies} are systems used to store knowledge from some domain, such as health or film. They are based in classes, their properties and the relationships among them. The Description Logics, subset of First Order Logics, are used to represent ontologies, since they have the formalism needed for it.

The inclusion of a new information, that is inconsistent with the knowledge base, is something very common to happen. In such cases, the new knowledge conflicts with parts of the old knowledge. The repair of the ontology becomes something necessary, in order to restore its consistency. Belief Revision techniques are used to achieve that.

A system used to build and use ontologies is called Protégé. One can define classes, axioms and relationships of a ontology with it. With the help of some plug-ins, some inferences can be made, and even inconsistencies can be found.

On this work, a theorical study is made, starting with the ontologies' origin, passing through Description Logics and ending with Belief Revision. Furthermore, using an already build plug-in and several other implementations, a program that executes some Belief Revision operations has been made.

\noindent \textbf{Keywords:} Belief Revision, OWL, Description Logics, Ontologies, Knowlege Representation.