\chapter*{Resumo}

\lettrine{A}{s} ontologias são sistemas usados para representar algum conhecimento de algum do\-mí\-nio, como a saúde ou o cinema. Elas são baseadas em classes, propriedades das classes e relações entre elas. O termo importado da Filosofia tem grande utilidade na Com\-pu\-ta\-ção.

As Lógicas de Descrição, que são sublinguagens da Lógica de Primeira Ordem, têm vários usos. Um deles é a construção de ontologias, já que representam o formalismo necessário para tal. 

Pode acontecer, em algum dado momento, que a inclusão de algum novo conhecimento torne a base de dados já existente inconsistente, ou seja, o novo conhecimento entra em conflito com algum que já estava lá. Nestes casos, é preciso reparar a ontologia, restaurando a sua consistência. Técnicas de Revisão de Crenças podem ser usadas para isso.

Um sistema utilizado para construir e usar ontologias é o \textit{Protégé}. Com ele se definem as classes, axiomas e relações da ontologia. Com o auxílio de \textit{plug-ins}, é possível fazer inferências e até encontrar as inconsistências. 

Nesse trabalho é feito um estudo teórico que vai desde as origens das ontologias, passando pelas Lógicas de Descrição e chegando, enfim, à área de Revisão de Crenças. Além disso, a partir de um \textit{plug-in} construído e de diversas outras implementações, é construído um programa que implementa algumas operações de Revisão de Crenças.

\noindent \textbf{Palavras-chave:} Revisão de Crenças, OWL, Lógicas de Descrição, Ontologias, Informação, Representação de Conhecimento.