
\chapter{Revisão de Crenças}

\lettrine{A}{} área de pesquisa que trata do reparo de uma ontologia quando ela fica inconsistente é conhecida como Revisão de Crenças, que dá nome a este capítulo. Como escrito anteriormente, uma ontologia (ou um sistema de crenças) fica inconsistente quando alguma informação $ \alpha $, que é incompatível com o conhecimento já existente, chega até a base de conhecimento estudada. 

De uma forma geral, esse campo da pesquisa de Inteligência Artificial estuda qualquer alteração de estados epistêmicos, desde a simples adição de algum novo conhecimento que não entra em conflito com o que já está na ontologia. Além disso, ele lida também com a remoção segura de alguma informação, ou seja, quando ela é removida, não pode ser inferida.

O estado epistêmico de um agente nada mais é do que o conjunto de tudo o que ele acredita e como elas se relacionam num certo instante. Pode-se entender um estado epistêmico também como uma representação idealizada do estado cognitivo de um agente em determinado momento, como explicou Gärdenfors \cite{revisaoGardenfors}.

Uma alteração do estado epistêmico seria, portanto, uma revisão que acontece quando o agente recebe uma nova informação que entra em choque com as informações que ele possui no estado epistêmico atual. Essa revisão deve manter as crenças antigas ao máximo, fazendo assim, uma mudança mínima.

O paradigma AGM, que será usada neste trabalho, recebe este nome por causa dos autores do artigo considerado o pontapé inicial desta área de pesquisa \cite{revisaoAGM}. Nela, os estados de crenças são representados por conjuntos logicamente fechados de sentenças, ou seja, conjuntos $ K $ tais $ K = \text{Cn}(K) $, onde $ \text{Cn}(K) $ representa o conjunto de todas as consequências lógicas que $ K $. Quando $ K = \text{Cn}(K) $, diz-se que há um equilíbrio dos estados epistêmicos. Com $ K $ sendo logicamente fechado, se $ K \vdash \psi $, sendo $ \psi $ uma sentença qualquer, tem-se que $ \psi \in K $

As sentenças citadas acima são de uma lógica $ (L, \text{Cn}) $, tal que $ L $ é uma linguagem fechada em relação aos conectivos lógicos $ \land $, $ \lor $, $ \to $ e $ \lnot $ e que satisfaz:

\begin{description}
	\item[tarskianicidade] a lógica é monotônica (todas as consequências dedutíveis continuam assim mesmo após a adição de alguma sentença que não interfere nessa dedução), idempotente e satisfaz inclusão;
	\item[dedução] $ \alpha \in \text{Cn}(K) $ se e somente se $ \beta \to \alpha \in \text{Cn}(K) $, onde $ \alpha $ e $ \beta $ são sentenças lógicas;
	\item[compacidade] se $ \alpha \in Cn(K) $, então existe $ K’ \subseteq K $ finito tal que $ \alpha \in \text{Cn}(K’) $;
	\item[supraclassicalidade] toda consequência da lógica $ (L, \text{Cn}) $ é também uma consequência da lógica proposicional.
\end{description} 

Para verificar que a área de Revisão de Crenças possui seus estudos válidos, seja o seguinte exemplo do assunto da ontologia descrita neste trabalho, assumindo que ela possui os fragmentos de conhecimento abaixo em alguma linguagem formal de representação, e que ela está consistente e equilibrada, epistemicamente.

\begin{enumerate}
	\item \textit{Toda música brasileira pertence ao subgênero MPB;}
	\item \textit{Todos os cantores brasileiros cantam músicas brasileiras;}
	\item \textit{Ivete Sangalo é uma cantora brasileira;}
	\item \textit{Ivete Sangalo canta a música "Sorte Grande".}
\end{enumerate}

Com esse conjunto de informações, é possível inferir que a música "Sorte Grande" pertence ao subgênero MPB. No entanto, suponha que a seguinte informação chegue até o agente:

\begin{center}
	\textit{A música "Sorte Grande" pertence ao subgênero Axé.}
\end{center}

Como pode-se ver, a nova informação entra em choque direto com a inferência realizada. As maneiras de reparar a ontologia e qual delas escolher para fazê-lo serão descritas nesse capítulo.

\section{Tratamento da informação}

Antes de ver como é possível tratar as inconsistências causadas pela entrada de alguma informação nova, é necessário observar como uma sentença $ \alpha $ qualquer é tratada em algum conjuntos de crenças $ K $.

Existem três tipo de tratamento, a serem descritos abaixo:

\begin{enumerate}
	\item $ \alpha $ é aceita pelo conjunto de crenças. Isso pode acontecer de duas maneiras diferentes:
	\begin{enumerate}
		\item $ \alpha $ é aceita explicitamente. Quando isso acontece, temos que $ \alpha \in K $;
		\item $ \alpha $ é aceita implicitamente. Esse caso ocorre quando $ \alpha \in \text{Cn}(K) \setminus K $.
	\end{enumerate}
	\item $ \alpha $ é rejeitada. Aqui, temos que $ \lnot \alpha \in K $. 
	\item $ \alpha $ é indeterminada. Deste modo, $ K $ não possui conhecimento sobre a sentença $ \alpha $, assim, $ \lnot \alpha \notin K $ e $ \alpha \notin K $.
\end{enumerate}

Existem também algumas questões metodológicas que precisam ser resolvidas, ou pelo menos observadas antes de alguma revisão de crenças. Gärdenfors \cite{revisaoGardenfors2} definiu algumas delas.

A primeira é em relação à representação das crenças na base de dados. Vale notar que, como definido no capítulo 2, é necessário o uso de alguma linguagem formal de representação. A maioria das bases de dados trabalha com fatos e regras como formas primitivas de informação. O mecanismo escolhido para a revisão deve levar em conta o formalismo escolhido para a representação.

Uma outra questão se preocupa com os elementos explicitamente representados na base de crenças e os que podem ser derivados desses. Existem bases que dão algum \textit{status} especial aos elementos explícitos, e outras que dão a mesma importância para todos. 

Neste trabalho, apenas os casos que a aceitação de $ \alpha $, pois serão estudados apenas sistemas de crenças logicamente fechados (também conhecidos como bases de crenças), onde vale o equilíbrio epistêmico. O uso de bases de crenças aumenta a eficiência do reparo \cite{revisaoHansson}.  

A última questão é referente a qual retração, ou seja, qual edição fazer na base de dados. A lógica, propriamente dita, não é suficiente para definir quais são os melhores elementos para serem removidos ou mantidos. Uma das ideias referentes a isso é que a quantidade informação perdida durante o reparo seja a mínima possível. 

Nas bases de dados que dão diferentes prioridades para as suas crenças, pode-se remover as que possuem menor prioridade, em detrimento daquelas com maior importância. Tudo isso depende de como a base de crenças está estruturada.

\section{Operações clássicas de Revisão de Crenças}

Existem três operações clássicas de Revisão de Crenças. São elas a expansão, a revisão e a contração. Apenas a expansão é definida diretamente, enquanto as duas últimas são definidas por postulados. 

Para as definições que serão feitas será usada uma lógica $ L $, que é baseada na Lógica de Primeira Ordem. As variáveis, que são as sentenças lógicas, serão representadas pelo alfabeto grego. Os conjuntos de crenças, com letras latinas maiúsculas.

\subsection{Expansão}

A única operação definida diretamente recebe a notação $ K + \alpha $. Ela é também a mais simples, já que representa a chegada de algum conhecimento à base sem que os anteriores sofram modificações. Além disso, as crenças que podem ser inferidas também são adicionadas.

O conjunto resultante é, portanto, formado pelo fecho lógico da união do conjunto inicial com a informação nova, o que significa que $ K + \alpha = \text{Cn}(K + \alpha) $. Vale ressaltar que o conjunto deve permanecer consistente.

A informação $ \alpha $ era indeterminada. Depois da operação de expansão, tal conhecimento passa a ser aceito, se $ \alpha $ é consistente com $ K $.

\subsection{Contração}

Esta operação seria o contrário da anterior. Em vez de uma nova informação ser adicionada à base de conhecimento, deseja-se que algum conhecimento seja removido. A operação é caracterizada por $ K - \alpha $. No entanto, as analogias com a expansão acabam por aí.

Como a consistência é um resultado desejável, às vezes, só retirar $ \alpha $ de $ K $ não é suficiente. Nesses casos, é necessário desistir de algumas crenças do conjunto. As sentenças que são abandonadas são aquelas que implicam $ \alpha $. O critério da mudança mínima deve ser aplicado aqui, removendo o mínimo de sentenças possíveis, ou as que possuem prioridade menor.

A informação $ \alpha $, que era aceita, agora é indeterminada. Agora, o conjunto resultante $ K - \alpha $ não possui $ \alpha $ explicitamente nem implicitamente. 

Os postulados de racionalidade que regem a contração seguem abaixo, com uma sucinta explicação sobre o seu significado.

\begin{description}
	\item[Fecho] $ K - \alpha = \text{Cn}(K - \alpha) $ \\ Após a contração, o fecho lógico do conjunto é mantido.
	\item[Inclusão] $ K - \alpha \subseteq K$ \\ A operação não permite que novas sentenças sejam adicionadas ao conjunto, resultando em um conjunto com o mesmo número de sentenças ou menos.
	\item[Vacuidade] Se $ \alpha \notin K $, então $ K - \alpha = K $ \\ Quando a fórmula que se deseja realizar a contração não está no conjunto, o mesmo é inalterado.
	\item[Sucesso] Se $ \alpha \notin \text{Cn}(\varnothing) $, então $ \alpha \notin K - \alpha $ \\ A não ser que $ \alpha $ seja uma tautologia, ela não estará contida no conjunto resultante da operação.
	\item[Equivalência] Se $ \text{Cn}(\alpha) = \text{Cn}(\beta) $, então, $ K - \alpha = K - \beta $ \\ Se duas fórmulas possuem consequências lógicas iguais (são logicamente equivalentes), a contração por uma terá o mesmo resultado do que a contração pela outra, embora $ \alpha \neq \beta $, em alguns casos.
	\item[Recuperação] $ K \subseteq (K - \alpha) + \alpha $ \\ A operação de contração é desfeita pela expansão.
	\item[Intersecção conjuntiva] $ (K - \alpha) \cap (K - \beta) \subseteq K - (\alpha \land \beta) $ \\ As crenças que não foram descartadas na contração por $ \alpha $ ou por $ \beta $ serão mantidas na contração por $ \alpha \land \beta $.
	\item[Inclusão conjuntiva] Se $ \alpha \notin K - (\alpha \land \beta) $ então, $ K - (\alpha \land \beta) \subseteq K - \alpha$ \\ Se a sentença $ \alpha $ é deletada na contração por $  (\alpha \land \beta )$, então tudo o que seria removido na contração por $ \alpha $ apenas, será removido. 
\end{description}

Os seis primeiros postulados são os chamados postulados básicos. Os últimos dois são os postulados suplementares.

Ainda para a contração existem dois construtores especiais para essa operação, que serão discutidos mais à frente.