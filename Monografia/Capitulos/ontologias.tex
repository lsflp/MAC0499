\chapter{Ontologias}

\lettrine{A}{} palavra \textit{Ontologia} veio do grego, assim como vários outros termos que se referem à áreas de estudo. Seu significado, no entanto, é muito mais abstrato. Diferentemente de \textit{Biologia}, que é "o estudo da vida", a palavra cujo plural dá nome a esse capítulo quer dizer "o estudo do ser enquanto ser". O dicionário Merriam-Webster \cite{ontoMerriam} estende essa definição como: "um ramo da metafísica preocupado com a natureza e as relações do ser".

Importado da Filosofia, esse conceito começa a ser trabalhado muito antes da época dos computadores. Aristóteles já estudava Ontologia em suas Categorias \cite{ontoDahlberg}. No entanto, apenas em 1606, com o livro Ogdoas Scholastica, de Jacob Lorhard, foi que a palavra em si realmente surgiu. Esse termo ficou popular em 1729 com o livro Philosophia Prima sive Ontologia, de Christian Wolff, com a definição "\textit{Ontology or First Philosophy is the science of Being in general or as Being}".

Para a Ciência da Computação, a definição é um pouco diferente, embora possua muita semelhança com o conceito já explícito. Guarino \cite{ontoGuarino} definiu ontologia como "um artefato de engenharia, constituído por um vocabulário específico usado para descrever uma certa realidade, mais uma série de pressupostos explícitos acerca do significado que se atribui a esse vocabulário". Logo, uma ontologia seria uma reunião de sentenças lógicas que exibem alguma informação sobre o mundo.

Fazendo um paralelo entre ambas as disciplinas, pode-se observar que, enquanto a primeira existe um estudo sistemático da existência, na segunda existe um foco maior em o que pode ser representado.

As ontologias denotam uma especificação explícita de uma conceitualização \cite{ontoGruber}, e, uma vez construídas, permitem comunicação, compartilhamento e reúso de conhecimentos.
