\documentclass[12pt, a4paper]{book}
\usepackage{amsmath,amsthm,amsfonts,amssymb,amscd}
\usepackage[table]{xcolor}
\usepackage[margin=2cm]{geometry}
\usepackage{graphicx}
\usepackage{multicol}
\usepackage{mathtools}
\usepackage{palatino}
\usepackage[utf8]{inputenc}
\usepackage[brazil]{babel}
\usepackage[pdftex]{hyperref}
\usepackage{lettrine}
\usepackage{float}
\usepackage{caption}
\usepackage{subcaption}
\usepackage{algpseudocode}
\usepackage{inconsolata}
\usepackage{mathpazo}
\usepackage{setspace}
\usepackage[all]{hypcap}                    % soluciona o problema com o hyperref e capitulos
\usepackage[round,sort]{natbib} % citação bibliográfica textual(plainnat-ime.bst)
\usepackage{emptypage}  % para não colocar número de página em página vazia
\fontsize{60}{62}\usefont{OT1}{cmr}{m}{n}{\selectfont}

% ---------------------------------------------------------------------------- %
% Cabeçalhos similares ao TAOCP de Donald E. Knuth
\usepackage{fancyhdr}
\pagestyle{fancy}
\fancyhf{}
\renewcommand{\chaptermark}[1]{\markboth{\MakeUppercase{#1}}{}}
\renewcommand{\sectionmark}[1]{\markright{\MakeUppercase{#1}}{}}
\renewcommand{\headrulewidth}{0pt}

% ---------------------------------------------------------------------------- %

\newcommand\kcont{\protect\mathpalette{\protect\independenT}{\perp}}
\def\independenT#1#2{\mathrel{\rlap{$#1#2$}\mkern2mu{#1#2}}}

\newcommand{\krev}{\downarrow\downarrow}

\newlength{\tabcont}
\setlength{\parskip}{0.05in}

\urlstyle{same} 

\begin{document}
	
	\frontmatter 
	% cabeçalho para as páginas das seções anteriores ao capítulo 1 (frontmatter)
	\fancyhead[RO]{{\footnotesize\rightmark}\hspace{2em}\thepage}
	\setcounter{tocdepth}{2}
	\fancyhead[LE]{\thepage\hspace{2em}\footnotesize{\leftmark}}
	\fancyhead[RE,LO]{}
	\fancyhead[RO]{{\footnotesize\rightmark}\hspace{2em}\thepage}
	
	\onehalfspacing  % espaçamento
	
	% ---------------------------------------------------------------------------- %
	% CAPA
	\thispagestyle{empty}
	\begin{center}
		\vspace*{2.3cm}
		Universidade de São Paulo\\
		Instituto de Matemática e Estatística\\
		Bacharelado em Ciência da Computação
		
		
		\vspace*{3cm}
		\Large{Luís Felipe de Melo Costa Silva}
		
		
		\vspace{3cm}
		\textbf{Revisão de Crenças em Lógica de Descrição}
		
		
		\vskip 5cm
		\normalsize{São Paulo}
		
		\normalsize{Novembro de 2018}
	\end{center}
	
	% ---------------------------------------------------------------------------- %
	% Página de rosto
	%
	\newpage
	\thispagestyle{empty}
	\begin{center}
		\vspace*{2.3 cm}
		\textbf{\Large{Revisão de Crenças em Lógica de Descrição}}
		\vspace*{2 cm}
	\end{center}
	
	\vskip 2cm
	
	\begin{flushright}
		Monografia final da disciplina \\
		MAC0499 -- Trabalho de Formatura Supervisionado.
	\end{flushright}
	
	\vskip 5cm
	
	\begin{center}
		Supervisora: Profª. Drª. Renata Wassermann
		
		\vskip 5cm
		\normalsize{São Paulo}
		
		\normalsize{Novembro de 2018}
	\end{center}
	\pagebreak
	
	\pagenumbering{roman}     % começamos a numerar 
	
	\chapter*{Resumo}
	\chapter*{Abstract}
	
	\tableofcontents    % imprime o sumário
	
	\mainmatter
	
	\fancyhead[RE,LO]{\thesection}
	
	\chapter{Introdução}
	
	\chapter{Ontologias}
\label{chap:ontologias}

\lettrine{A}{} palavra Ontologia veio do grego, assim como vários outros termos que se referem a alguma área de estudo. Seu significado, no entanto, é muito mais abstrato. Diferentemente de Biologia, que é ''o estudo da vida'', a palavra cujo plural dá nome a este capítulo quer dizer ''o estudo do ser enquanto ser'. O dicionário Merriam-Webster\footnote{\url{http://www.merriam-webster.com}} estende essa definição como: "um ramo da metafísica preocupado com a natureza e as relações do ser".

Importado da Filosofia, esse conceito começa a ser trabalhado muito antes da época dos computadores. Aristóteles já estudava Ontologia em suas Categorias \citep{ontoDahlberg}. No entanto, apenas em 1606, com o livro \textit{Ogdoas Scholastica}, de Jacob Lorhard, foi que a palavra em si realmente surgiu. Esse termo ficou popular em 1729 com o livro \textit{Philosophia Prima: sive Ontologia}, de Christian Wolff, com a definição "\textit{Ontology or First Philosophy is the science of Being in general or as Being}" \footnote{"Ontologia ou Filosofia Primeira é a ciência do ser em real ou como um ser."}  \citep{ontoNickles}.

Para a Ciência da Computação, a definição é um pouco diferente, embora possua muita semelhança com o conceito já explícito. Guarino definiu ontologia como "um artefato de engenharia, constituído por um vocabulário específico usado para descrever uma certa realidade, mais uma série de pressupostos explícitos acerca do significado que se atribui a esse vocabulário" \citep{ontoGuarino}. Logo, uma ontologia é uma reunião de sentenças lógicas que exibem alguma informação sobre alguma área do mundo para resolver algum problema relacionado a ela.

Fazendo um paralelo entre ambas as disciplinas, pode-se observar que, enquanto a primeira faz um estudo sistemático da existência, na segunda existe um foco maior no que pode ser representado.

As ontologias são estudadas na área de Inteligência Artificial, que está preocupada com a automação do comportamento inteligente. Na prática, elas funcionam como um sistema ''\textit{tell and ask}'' \citep{ontoRussel}. Algumas coisas são contadas para os agentes inteligentes (uma entidade autônoma com comportamento que simula inteligência), e então, perguntas podem ser feitas para eles, embora não precisem saber todas as respostas. 

Elas surgiram de um contexto onde os cientistas desejam modelar e representar o mundo para as máquinas, e isso ocorre desde a origem dos computadores. Como cada pessoa possui uma visão de mundo, cada modelagem será diferente de algum jeito. Por isso, a construção de ontologias é um tópico que merece estudo.

As ontologias denotam uma "especificação explícita de uma conceitualização"  \citep{ontoGruber}, e, uma vez construídas, permitem comunicação, compartilhamento e reúso de conhecimentos.

\section{Conceitualização}
	
Existe um certo debate sobre a definição de Conceitualização. Depois de propor o que era uma ontologia, Gruber sugeriu que uma Conceitualização "é uma visão abstrata e simplificada do mundo que se quer representar para algum propósito". Essa definição parece boa, mas deixa algumas pontas soltas. O que seria uma "visão", por exemplo, deixa algumas dúvidas.

Para Guarino e Giaretta, uma Conceitualização pode ser entendida como “uma estrutura semântica intensional que codifica as regras implícitas que determinam a estrutura de uma porção da realidade” \citep{ontoGiaretta}. Essa definição é um pouco mais concreta, e já é possível pensar sobre ela computacionalmente.

Pode-se inferir que uma Conceitualização é uma modelagem de parte de algum domínio do conhecimento. O domínio nada mais seria do que alguma disciplina, como a Geografia, Música, Enologia, entre outros. Tal modelagem é feita a partir de alguma linguagem formal de representação (em geral, Lógicas de Descrição) e deve levar em conta a generalidade que se aplica ao domínio escolhido.

Vale lembrar que, embora sejam amplamente utilizadas na vida real, as linguagens naturais (como a língua portuguesa e a língua inglesa) não são consideradas linguagens formais. Isso tem uma explicação simples. Basta lembrar das figuras de linguagem, como a metáfora e o eufemismo, amplamente usadas nos discursos escrito e falado. 

\section{Construindo uma ontologia}

Para construir uma ontologia, é necessário escolher um domínio e o nível de generalidade que é necessário que ela atinja. Também deve-se ter em mente quem vai usá-la. Para que ela alcance o máximo de utilidade, é necessário que as perguntas que se deseja que ela responda sejam feitas antes de sua construção.

Geralmente, grandes ontologias são projetadas por equipes interdisciplinares, para que ela seja o mais correta e abrangente quanto possível. Quando se confecciona uma ontologia, é necessário que sejam feitas algumas decisões de projeto.  

Gruber fez uma proposta de critérios de \textit{design} para ontologias com o objetivo de tornar o compartilhamento de conhecimento e interoperabilidade com programas baseados em conhecimento mais fácil \citep{ontoGruber}. Eles são os seguintes:

\begin{description}
	\item[Clareza] Uma ontologia deve ter uma linguagem clara e efetiva na definição de seus termos. Tal definição deve ser objetiva. Embora ela possa vir de situações sociais ou requisitos computacionais, ela deve ser independente destes contextos. Usar uma linguagem lógica, é um meio para este fim, ou seja, quando for possível fazer uma definição usando axiomas lógicos, isso deve ser feito. Onde possível, uma definição completa (com condições necessárias e suficientes) é preferível a uma definição parcial (com condições necessárias ou suficientes). Todas devem ser documentadas usando linguagem natural.
	\item[Coesão] Uma ontologia deve permitir inferências consistentes com suas definições. No mínimo, os axiomas usados nas definições devem ser logicamente consistentes. A coerência também deve ser aplicada aos conceitos informais, definidos na linguagem natural da documentação e nos exemplos. Se uma sentença que pode ser inferida contradiz uma definição ou exemplo dado informalmente, a ontologia não é coesa.
	\item[Estendibilidade] Uma ontologia deve ser projetada para ser capaz de antecipar o uso de conhecimento compartilhado. Ela deve oferecer uma fundação conceitual de modo que o novo conhecimento possa ser apoiado nela e que a ontologia possa ser estendida e especializada, ou seja, novos termos podem ser definidos usando o vocabulário existente, de modo que a revisão das crenças do conhecimento anterior possa ser evitada ao máximo.
	\item[Viés mínimo de codificação] A conceitualização deve ser especificada num nível de conhecimento que não dependa de uma codificação que utiliza um nível de símbolos particulares. Um viés de codificação ocorre quando as escolhas de representação são feitas por pura conveniência de notação ou de implementação. Como agentes de conhecimento podem ser implementados em diferentes sistemas e estilos de representação, isso deve ser minimizado.
	\item[Compromisso ontológico mínimo] Isso faz com que ela suporte as atividades de compartilhamento de conhecimento desejadas. Quanto menos suposições sobre o mundo modelado, melhor. Isso permite que as partes comprometidas com a ontologia sejam livres para especializar e instanciar a ontologia o quanto quiserem. Já que isso é baseado no uso consistente do vocabulário, ele pode ser minimizado usando uma teoria fraca (genérica) e definindo apenas os conceitos necessários para a comunicação do conhecimento consistente com esta teoria.
\end{description}

É possível notar que todos esses critérios poderão não ser atendidos ao mesmo tempo, portanto, alguns \textit{trade-offs} deverão ser feitos.
Podemos ter um conflito, por exemplo, entre os critérios \textbf{Clareza} e \textbf{Estendibilidade}, já que o máximo de clareza implica que as definições terão a sua interpretação restrita.

\subsection{Componentes de uma ontologia}

Computacionalmente falando, as ontologias possuem cinco partes. Para ilustrá-las melhor, vamos fazer uma ontologia sobre Música. Será usada uma fonte monoespaçada para as partes dela.

\begin{description}
	\item[Classes] Descrevem os conceitos de um certo domínio do discurso. São o foco das ontologias. São uma coleção de todos os particulares aos quais é possível aplicar um termo geral. Por exemplo: a classe \texttt{Cantor}.
	\item[Propriedades] São atributos que descrevem características do conceito a que uma classe se refere. Em nosso exemplo, a classe \texttt{Pessoa} possui as propriedades \texttt{Nome} e \texttt{Idade}. A classe \texttt{Cantor} pode ter a propriedade \texttt{RitmoPredominante}.
	\item[Relações] Ontologias são constituídas de relações hierárquicas. Por exemplo \texttt{Cantor} $ \to $ \texttt{Pes\-so\-a}. A hierarquia de classes representa uma relação “\textit{is-a}” \citep{ontoFranca}. Tais relações são transitivas. Existem outras relações, como \texttt{Cantor canta Musica} e \texttt{Compositor compoe Musica}.
	\item[Restrições] São os tipos das propriedades, por exemplo: enquanto para \texttt{Nome} e \texttt{Ritmo\-Pre\-do\-mi\-nan\-te} uma \textit{string} seja suficiente, para a \texttt{Idade}, um inteiro pode ser utilizado. Além disso, as restrições podem ser referentes às classes. Exemplo: a relação \texttt{canta} tem domínio \texttt{Cantor.}
	\item[Instância] é um indivíduo (por exemplo, \texttt{MariahCarey}), de uma classe (como \texttt{Cantor}, \texttt{Com\-po\-si\-tor} ou \texttt{Pessoa}). Vale lembrar que uma instância não é uma subclasse.
\end{description}

Uma classe pode ter nomes diferentes em certo idiomas (por exemplo, \texttt{Cantor} e \texttt{Singer} seriam a mesma classe). Isso não é um problema pois não é o nome que define uma classe. Sinônimos e palavras em línguas distintas não representam classes diferentes.

Várias classes subordinadas a uma superclasse são consideradas irmãs. Elas devem ter o mesmo nível de generalidade. Por exemplo, seja uma classe \texttt{Musica}. Suas subclasses podem ser \texttt{Cancao}, \texttt{MusicaArgentina} e \texttt{MusicaBrasileira}. Tais classes são irmãs.

Existem três processos para definir as classes e a hierarquia, a seguir:

\begin{description}
	\item[\textit{Top-down}] Vai das classes mais genéricas para as mais específicas. Um exemplo seria criar as classes \texttt{Musica}, \texttt{Pessoa}, entre outras.
	\item[\textit{Bottom-up}] Vai das classes mais específicas para as mais genéricas, agrupando as específicas já criadas. Na ontologia estudada aqui, é possível começar de músicas conhecidas para então, separá-las por ritmos.
	\item[Vai-e-Vem] Define conceitos simples para generalizá-los e especificá-los. É uma com\-bi\-na\-ção dos dois primeiros itens.
\end{description}

\subsection{Montando a ontologia}

Montar uma ontologia é um processo que segue os seguintes passos:

\begin{itemize}
	\item Definir as classes da ontologia.
	\item Colocá-las em uma hierarquia taxonômica.
	\item Determinar suas propriedades e restrições.
	\item Criar uma base de conhecimento para essas classes e propriedades, ou seja, preencher a ontologia com as instâncias.
	\item Colocar os valores das propriedades para as instâncias.
\end{itemize}

Embora pareça ser direto, esse processo é iterativo, como afirmam Noy e McGuinness  \citep{ontoNoy}. Uma vez feito, deve ser repetido para que haja uma adequação das classes com as instâncias colocadas, pois, por exemplo, se uma classe acabar com apenas uma subclasse, a modelagem pode ter um problema, ou a ontologia não está completa. E ainda, se uma classe possui mais de uma dúzia de subclasses, novas categorias (classes ou subclasses) podem ser necessárias. 

Às vezes, uma classe possui muitas propriedades específicas e diferentes em várias de suas instâncias. Nesse caso, a inserção de uma classe deixará a ontologia mais compreensível. A nova classe fará com que a distinção das instâncias ocorra, efetivamente, evitando mal-entendidos.

Teoricamente, o processo nunca acaba. Na prática, ele é interrompido quando a ontologia fornece respostas suficientemente boas para a maior parte das consultas realizadas, ou seja, tal critério é subjetivo.

Ainda em relação às classes, algumas observações podem ser feitas. A primeira é em relação à herança múltipla. Ela acontece quando uma classe é subclasse de várias outras classes, por exemplo, a classe \texttt{PopBrasileiro}, pode pertencer à classe \texttt{Pop} e à classe \texttt{MusicaBrasileira}. Isso é aceitável, pois no mundo real, acontece várias vezes e de diversas maneiras.

As ontologias podem representar classes disjuntas. Classes disjuntas são aquelas que não possuem uma intersecção. Em nossa ontologia, \texttt{Cantor} e \texttt{Compositor} não são disjuntas. No entanto, \texttt{Cancao} e \texttt{HinoNacional} são classes disjuntas.

Para os nomes das classes, não há uma convenção específica, só há um consenso de que manter um padrão de nomenclatura é algo bom. Para fazer isso, pode-se usar \texttt{snake\textunderscore case}, \texttt{camelCase} ou usar espaços na grafia.

Uma ontologia não precisa ter toda a informação existente sobre o domínio. Não é necessário especializar ou generalizar mais do que seja necessário para a aplicação. Além disso, as classes não precisam ter todas as propriedades possíveis e nem carregar todas as distinções que estão no mundo. Isso significa que a ontologia deve ser o modelo mais simples para o problema que se deseja resolver.

Algumas relações podem ter uma inversa, assim como ocorre com funções matemáticas. Uma relação que possui uma inversa pode ser \texttt{Cantor canta Musica}, com sua inversa sendo \texttt{Musica cantadaPor Cantor}, e a relação \texttt{Compositor compoe Musica} pode ter a inversa \texttt{Musica compostaPor Compositor}.

\section{Problemas relacionados}

Existem alguns problemas relacionados a ontologias. Os mais comuns são os problemas de modelagem e construção. Um problema muito comum é a existência, na linguagem natural, de homônimos e sinônimos. Deve existir um cuidado especial com eles, já que eles podem levar a uma confusão na nomenclatura.

Em relação à modelagem, o problema de não utilizar uma equipe interdisciplinar especializada pode levar a uma cobertura incompatível de conceitos, ou seja, as classes podem ficar muito distantes da realidade. Além disso, usar fontes não confiáveis na ontologia pode fazer com que os problemas que elas estavam sendo feitas para resolver, não sejam resolvidos corretamente.

Tais problemas tornam-se muito maiores quando duas ontologias são integradas. A possível integração entre ontologias é um dos motivos para elas existirem, afinal, isso pode acelerar o seu desenvolvimento. Em nosso exemplo, se já existir uma ontologia sobre Música Brasileira, poderemos absorvê-la, mas o cuidado terá de ser redobrado em relação às questões acima. 

Em relação à implementação, os pontos que surgem são em geral a respeito da linguagem utilizada, mas isso será tratado no \autoref{chap:logicas}.

Além dos já citados, pode acontecer de chegar um novo conhecimento e a ontologia ficar inconsistente. Neste caso, terão de ser usadas algumas operações de Revisão de Crenças, foco deste trabalho, que será estudado no \autoref{chap:revisao}.

A ontologia desenvolvida nesse capítulo segue na \autoref{img:Esquema}.

\begin{figure}
	\centering
	\includegraphics[width=0.6\textwidth]{Capitulos/Ontologias/OntologiaMusica}
	\caption{Esquema de classes de uma ontologia sobre Música. As linhas tracejadas indicam as subclasses e as setas, as relações.}
	\label{img:Esquema}
\end{figure}
	
	\chapter{Lógicas de Descrição}

\lettrine{A}{} área de Inteligência Artificial é composta por diversas sub-áreas responsáveis por diversos estudos. Uma delas é a de Representação de Conhecimento, que cuida da construir formalismos adequados para expressar conhecimento sobre um domínio. Tal área tem seus estudos válidos, afinal, entidades inteligentes possuem algum tipo de conhecimento e precisam fazer inferências a partir dele. Para fazer isso, os formalismos devem possuir alguma representação.

Existe um consenso, já mencionado no capítulo anterior, de que linguagens formais são uma maneira boa de caracterizar axiomas lógicos para a modelagem de uma Ontologia. O principal formalismo utilizado para representar conhecimento, com um cuidado especial para as terminologias, são as Lógicas de Descrição (doravante, LD).

Essas lógicas são um subconjunto da Lógica de Primeira Ordem, que por sua vez, estende a Lógica Proposicional. As LD são utilizadas por sua expressividade. Para anotar que uma \texttt{Cancao} ou um \texttt{HinoNacional} são uma \texttt{Musica}, usando a ontologia que está sendo desenvolvida neste trabalho, poderíamos utilizar as seguintes sentenças, usando as lógicas até agora citadas:

\begin{table}[H]
	\centering
	\begin{tabular}{|l|l|l}
		\cline{1-2}
		Lógica                   & Sentença & \\ \cline{1-2}
		Proposicional     & \texttt{cancao} $\lor$ \texttt{hinoNacional} $\to$ \texttt{musica} & \\ \cline{1-2}
		de Primeira Ordem & $\forall x(\texttt{Cancao}(x) \lor \texttt{HinoNacional}(x) \to \texttt{Musica}(x))$         & \\ \cline{1-2}
		de Descrição      & \texttt{Cancao} $\sqcup$ \texttt{HinoNacional} $\sqsubseteq$ \texttt{Musica}  & \\ \cline{1-2}
	\end{tabular}
\caption{A mesma sentença expressa em diferentes lógicas}
\end{table}

Podemos observar que a notação da LD permite que a interpretação de suas sentenças seja feita usando a Teoria dos Conjuntos. Na última sentença da tabela, podemos inferir que a união dos conjuntos \texttt{Cancao} e \texttt{HinoNacional} está contida no conjunto \texttt{Musica}.

\section{Conceitos, papéis e indivíduos}

As LD possuem um jeito próprio de organizar o conhecimento. Elas utilizam de três definições para retratar o que é desejado. Cada uma delas é utilizada para a construção de uma ontologia. São, a seguir:

\begin{itemize}
	\item Conceitos: Também chamados de classes, representam um conjunto de indivíduos. Nas ontologias, são equivalentes às Classes.
	\item Papéis: Podem ser denotados como propriedades e retratam as relações binárias que existem entre os indivíduos. Basicamente, são as Relações de uma ontologia. Alguns papéis podem ter uma função de caracterização, ou seja, definindo alguns atributos para o conceito. Tal função corresponde a uma Propriedade de uma Classe, em uma ontologia.
	\item Indivíduos: Representam os particulares que existem dentro de um Conceito. Para as ontologias, são as Instâncias.
\end{itemize}

Logo, uma LD é definida por uma tripla $ <N_C, N_R, N_I> $, onde $ N_C $ é um conjunto de conceitos atômicos, $ N_R $ é um conjunto de papéis atômicos e $ N_I $ é um conjunto de nomes de indivíduos.

Feita essa distinção, é possível dividir o conhecimento em duas partes, com as LD. A primeira delas é a \textit{TBox}. Ela se refere ao conhecimento terminológico, ou seja, o conhecimento intensional do domínio. Nela ficam os conceitos, as propriedades e as restrições.

A outra parte é a \textit{ABox}, e nela fica o conhecimento assertivo, ou seja, o conhecimento extensional. Ela contém as asserções sobre as instâncias, ou seja, como eles se encaixam nas definições explícitas na \textit{TBox}.

\section{Equivalências com a Lógica de Primeira Ordem}

\section{Interpretação e Consequência Lógica}

\section{Outras linguagens}

\section{Tableau}
	
	
\chapter{Revisão de Crenças}

\lettrine{A}{} área de pesquisa que trata do reparo de uma ontologia quando ela fica inconsistente é conhecida como Revisão de Crenças, que dá nome a este capítulo. Como escrito anteriormente, uma ontologia (ou um sistema de crenças) fica inconsistente quando alguma informação $ \alpha $, que é incompatível com o conhecimento já existente, chega até a base de conhecimento estudada. 

De uma forma geral, esse campo da pesquisa de Inteligência Artificial estuda qualquer alteração de estados epistêmicos, desde a simples adição de algum novo conhecimento que não entra em conflito com o que já está na ontologia. Além disso, ele lida também com a remoção segura de alguma informação, ou seja, quando ela é removida, não pode ser inferida.

O estado epistêmico de um agente nada mais é do que o conjunto de tudo o que ele acredita e como elas se relacionam num certo instante. Pode-se entender um estado epistêmico também como uma representação idealizada do estado cognitivo de um agente em determinado momento, como explicou Gärdenfors \cite{revisaoGardenfors}.

Uma alteração do estado epistêmico seria, portanto, uma revisão que acontece quando o agente recebe uma nova informação que entra em choque com as informações que ele possui no estado epistêmico atual. Essa revisão deve manter as crenças antigas ao máximo, fazendo assim, uma mudança mínima.

O paradigma AGM, que será usada neste trabalho, recebe este nome por causa dos autores do artigo considerado o pontapé inicial desta área de pesquisa \cite{revisaoAGM}. Nela, os estados de crenças são representados por conjuntos logicamente fechados de sentenças, ou seja, conjuntos $ K $ tais $ K = \text{Cn}(K) $, onde $ \text{Cn}(K) $ representa o conjunto de todas as consequências lógicas que $ K $. Quando $ K = \text{Cn}(K) $, diz-se que há um equilíbrio dos estados epistêmicos. Com $ K $ sendo logicamente fechado, se $ K \vdash \psi $, sendo $ \psi $ uma sentença qualquer, tem-se que $ \psi \in K $

As sentenças citadas acima são de uma lógica $ (L, \text{Cn}) $, tal que $ L $ é uma linguagem fechada em relação aos conectivos lógicos $ \land $, $ \lor $, $ \to $ e $ \lnot $ e que satisfaz:

\begin{description}
	\item[tarskianicidade] a lógica é monotônica (todas as consequências dedutíveis continuam assim mesmo após a adição de alguma sentença que não interfere nessa dedução), idempotente e satisfaz inclusão;
	\item[dedução] $ \alpha \in \text{Cn}(K) $ se e somente se $ \beta \to \alpha \in \text{Cn}(K) $, onde $ \alpha $ e $ \beta $ são sentenças lógicas;
	\item[compacidade] se $ \alpha \in Cn(K) $, então existe $ K’ \subseteq K $ finito tal que $ \alpha \in \text{Cn}(K’) $;
	\item[supraclassicalidade] toda consequência da lógica $ (L, \text{Cn}) $ é também uma consequência da lógica proposicional.
\end{description} 

Para verificar que a área de Revisão de Crenças possui seus estudos válidos, seja o seguinte exemplo do assunto da ontologia descrita neste trabalho, assumindo que ela possui os fragmentos de conhecimento abaixo em alguma linguagem formal de representação, e que ela está consistente e equilibrada, epistemicamente.

\begin{enumerate}
	\item \textit{Toda música brasileira pertence ao subgênero MPB;}
	\item \textit{Todos os cantores brasileiros cantam músicas brasileiras;}
	\item \textit{Ivete Sangalo é uma cantora brasileira;}
	\item \textit{Ivete Sangalo canta a música "Sorte Grande".}
\end{enumerate}

Com esse conjunto de informações, é possível inferir que a música "Sorte Grande" pertence ao subgênero MPB. No entanto, suponha que a seguinte informação chegue até o agente:

\begin{center}
	\textit{A música "Sorte Grande" pertence ao subgênero Axé.}
\end{center}

Como pode-se ver, a nova informação entra em choque direto com a inferência realizada. As maneiras de reparar a ontologia e qual delas escolher para fazê-lo serão descritas nesse capítulo.

\section{Tratamento da informação}

Antes de ver como é possível tratar as inconsistências causadas pela entrada de alguma informação nova, é necessário observar como uma sentença $ \alpha $ qualquer é tratada em algum conjuntos de crenças $ K $.

Existem três tipo de tratamento, a serem descritos abaixo:

\begin{enumerate}
	\item $ \alpha $ é aceita pelo conjunto de crenças. Isso pode acontecer de duas maneiras diferentes:
	\begin{enumerate}
		\item $ \alpha $ é aceita explicitamente. Quando isso acontece, temos que $ \alpha \in K $;
		\item $ \alpha $ é aceita implicitamente. Esse caso ocorre quando $ \alpha \in \text{Cn}(K) \setminus K $.
	\end{enumerate}
	\item $ \alpha $ é rejeitada. Aqui, temos que $ \lnot \alpha \in K $. 
	\item $ \alpha $ é indeterminada. Deste modo, $ K $ não possui conhecimento sobre a sentença $ \alpha $, assim, $ \lnot \alpha \notin K $ e $ \alpha \notin K $.
\end{enumerate}

Existem também algumas questões metodológicas que precisam ser resolvidas, ou pelo menos observadas antes de alguma revisão de crenças. Gärdenfors \cite{revisaoGardenfors2} definiu algumas delas.

A primeira é em relação à representação das crenças na base de dados. Vale notar que, como definido no capítulo 2, é necessário o uso de alguma linguagem formal de representação. A maioria das bases de dados trabalha com fatos e regras como formas primitivas de informação. O mecanismo escolhido para a revisão deve levar em conta o formalismo escolhido para a representação.

Uma outra questão se preocupa com os elementos explicitamente representados na base de crenças e os que podem ser derivados desses. Existem bases que dão algum \textit{status} especial aos elementos explícitos, e outras que dão a mesma importância para todos. 

Neste trabalho, apenas os casos que a aceitação de $ \alpha $, pois serão estudados apenas sistemas de crenças logicamente fechados (também conhecidos como bases de crenças), onde vale o equilíbrio epistêmico. O uso de bases de crenças aumenta a eficiência do reparo \cite{revisaoHansson}.  

A última questão é referente a qual retração, ou seja, qual edição fazer na base de dados. A lógica, propriamente dita, não é suficiente para definir quais são os melhores elementos para serem removidos ou mantidos. Uma das ideias referentes a isso é que a quantidade informação perdida durante o reparo seja a mínima possível. 

Nas bases de dados que dão diferentes prioridades para as suas crenças, pode-se remover as que possuem menor prioridade, em detrimento daquelas com maior importância. Tudo isso depende de como a base de crenças está estruturada.

\section{Operações clássicas de Revisão de Crenças}

Existem três operações clássicas de Revisão de Crenças. São elas a expansão, a revisão e a contração. Apenas a expansão é definida diretamente, enquanto as duas últimas são definidas por postulados. 

Para as definições que serão feitas será usada uma lógica $ L $, que é baseada na Lógica de Primeira Ordem. As variáveis, que são as sentenças lógicas, serão representadas pelo alfabeto grego. Os conjuntos de crenças, com letras latinas maiúsculas.

\subsection{Expansão}

A única operação definida diretamente recebe a notação $ K + \alpha $. Ela é também a mais simples, já que representa a chegada de algum conhecimento à base sem que os anteriores sofram modificações. Além disso, as crenças que podem ser inferidas também são adicionadas.

O conjunto resultante é, portanto, formado pelo fecho lógico da união do conjunto inicial com a informação nova, o que significa que $ K + \alpha = \text{Cn}(K + \alpha) $.

A informação $ \alpha $ era indeterminada. Depois da operação de expansão, tal conhecimento passa a ser aceito.
	
	\chapter{Ferramentas Computacionais}

\lettrine{O}{} interesse no estudo de Desenvolvimento de Ontologias, Lógicas de Descrição e Revisão de Crenças, deve-se, em parte, às aplicações computacionais que tais áreas possuem. Neste capítulo, algumas delas serão abordadas.

\section{OWL}

O final da década de 1990 foi a época de surgimento da ideia de Web Semântica. Essa ideia, já consolidada hoje, representa uma extensão da \textit{World Wide Web} onde é possível o trabalho cooperativo entre as máquinas e os seres humanos. Ele ocorre a partir dos significados que são dados aos conteúdos disponíveis na rede para que ele seja compreensível por ambos os agentes \cite{ferramentasHerman}.

No início dos anos 2000, a \href{https://www.w3.org}{W3C} (sigla para Consórcio para a \textit{World Wide Web}) fundou o "\textit{Web Ontology Working Group}" \cite{ferramentasGrupo}, para que alguma Linguagem de Representação fosse criada. 

Em julho de 2002, os primeiros rascunhos foram publicados, e em fevereiro de 2004 \cite{ferramentasReco}, a OWL (\textit{Web Ontology Language}) tornou-se um padrão recomendado pela W3C para o processamento de ontologias.

Em outubro de 2007 \cite{ferramentasOWLGrupo2}, um novo grupo foi concebido para adicionar alguns recursos à linguagem. Dois anos depois, a W3C fez o lançamento da OWL 2, que seria compatível com editores e raciocinadores semânticos. Ela foi recomendada oficialmente pela W3C em dezembro de 2012 \cite{ferramentasOWLReco2}.

Abaixo temos algumas características das duas versões da OWL, que foi baseada em RDF e RDFS.

\subsection{OWL Web Ontology Language}

Possui três sublinguagens \cite{ferramentasOWL1}, cada uma com suas particularidades:

\begin{itemize}
	\item \textit{Lite}: Sublinguagem mais simples, feita para hierarquias de classificação com restrições simples. Suporta restrições de cardinalidade simples (apenas 0 ou 1).  Foi demonstrado que OWL-Lite é equivalente a $ \mathcal{SHIF^{(D)}} $.
	\item DL: É equivalente a $ \mathcal{SHOIN^{(D)}} $. Oferece expressividade máxima evitando alguns problemas de decidibilidade. Inclui todos os construtores da OWL. Possui mais complexidade formal do que a sublinguagem \textit{Lite}.
	\item \textit{Full}: Versão mais completa, oferecendo expressividade máxima e a liberdade sintática do RDF, só que sem garantias computacionais. Nessa sublinguagem, pode acontecer de uma Classe ser tratada como uma coleção de indivíduos ou como um indivíduo, o que acarreta problemas de decidibilidade. Não corresponde a uma lógica de descrição.
\end{itemize}

\subsection{OWL 2 Web Ontology Language}

Assim como a primeira versão, também possui sublinguagens \cite{ferramentasOWL2}. Elas são:

\begin{itemize}
	\item EL: Corresponde à lógica $ \mathcal{EL} $++. Funciona bem para ontologias com muitas classes e/ou propriedades. Permite a existência de algoritmos de inferência polinimiais.
	\item QL: Baseada na lógica de descrição DL-\textit{Lite}, foi feita para aplicações em que o volume de instâncias é muito grande e que a consulta de dados é a tarefa mais realizada. Tais consultas podem ser implementadas usando sistemas de bancos de dados convencionais.
	\item RL: Criada para ontologias que precisam de raciocínio escalável, ou seja, que são possivelmente grandes em número de instâncias, mas que precisam de uma sublinguagem que mantenha um bom poder expressivo. É baseada em uma lógica de descrição chamada DLP.
	\item DL: Assim como no lançamento anterior, oferece bastante expressividade, sem esbarrar em problemas de execução por causa da indecibilidade. Ela pode ser mapeada para a LD $ \mathcal{SROIQ^{(D)}} $. 
	\item \textit{Full}: Análoga à da primeira versão. Muito expressiva, porém, indecidível.
\end{itemize}

{\color{red} Colocar parte do arquivo owl da ontologia?}

\section{\textit{Protégé}}

O \href{https://protege.stanford.edu}{\textit{Protégé}} é um editor semântico, compatível com a OWL 2. Ele foi desenvolvido pelo Centro de Pesquisa para Informática Biomédica da Universidade de Stanford, em colaboração com alguns programadores da Universidade de Manchester \cite{ferramentasProtege}. Sua primeira versão foi lançada em 1999, e a atual é de 2017.

Seguindo os princípios do \textit{Software} Livre, ele é um arcabouço gratuito e de código aberto, feito para a construção de sistemas inteligentes, com o uso ou não de uma interface gráfica.

Assim como o ambiente de desenvolvimento integrado Eclipse, o \textit{Protégé} é bastante flexível porque é possível desenvolver uma grande variedade \textit{plug-ins} para serem acoplados a ele.

Para os exemplos dessa seção, será usado como exemplo parte da ontologia criada neste estudo:

\begin{itemize}
	\item Os conceitos: 
	\begin{itemize}
		\item \texttt{Musica} $ \equiv $ \texttt{Cancao} $ \sqcup $ \texttt{HinoNacional}.
		\item \texttt{Pop} $ \sqsubseteq $ \texttt{Cancao};
		\item \texttt{Pessoa} $ \equiv $ \texttt{Cantor} $ \sqcup $ \texttt{Compositor}.
	\end{itemize}
	\item As propriedades:
	\begin{itemize}
		\item \textbf{\texttt{canta}}, com domínio \texttt{Cantor} e contradomínio \texttt{Musica};
		\item \textbf{\texttt{cantadaPor}}, inversa da propriedade acima.
	\end{itemize}
	\item As instâncias e asserções:
	\begin{itemize}
		\item \texttt{MarinaAndTheDiamonds}, instância da classe \texttt{Cantor};
		\item \texttt{Primadonna}, instância da classe \texttt{Pop};
		\item \texttt{MarinaAndTheDiamonds \textbf{canta} Primadonna}.
		\item \texttt{Froot}, instância da classe \texttt{Pop};
		\item \texttt{MarinaAndTheDiamonds \textbf{canta} Froot}.	
	\end{itemize}
\end{itemize}

\subsection{Raciocinadores}

Um aspecto interessante sobre esse arcabouço é o uso de plug-ins de \textit{raciocínio}, como o {\color{red} Colocar o que utilizei.}. A partir deles, é possível fazer inferências. 

{\color{red} Colocar o exemplo da inferência que Froot é cantada por Marina and the Diamonds.}

\subsection{Buscas}

A partir de uma ontologia e de uma base de dados, é possível fazer buscas no arcabouço, utilizando, entre várias linguagens de busca, o SPARQL, por exemplo.

SPARQL é um acrônimo recursivo para \textit{SPARQL Protocol and RDF Query Language} \cite{ferramentasSPARQL}. Tem uma sintaxe que lembra a do SQL, e com seu uso, é possível tratar a ontologia como uma base de dados.  

Vale notar que a busca funciona apenas com a terminologia e as asserções gravadas no arquivo da ontologia. Para que as buscas funcionem sobre as inferências, é necessário exportá-las para um novo arquivo.

{\color{red} Colocar o exemplo de quais são as músicas pop que a Marina canta.}

{\color{red} Colocar o logo do Protégé.}
	
	\chapter{Implementação do \textit{plug-in}}
\label{chap:implementacao}

\lettrine{O}{} \textit{plug-in} construído para este trabalho é, na verdade, uma reunião de implementações realizadas em trabalhos anteriores. As operações cobertas pelo \textit{plug-in} e suas respectivas construções anteriores são:

\begin{description}
	\item[Contração] O \textit{plug-in} possui os dois construtores descritos neste trabalho, as Contrações \textit{Partial Meet} e a \textit{Kernel}. Ambas foram baseadas em implementações de algoritmos feitas na tese de doutorado de Cóbe  \citep{revisaoCobe}.
	\item[Revisão \textit{Kernel}] Para essa operação, os códigos feitos por Resina para o seu trabalho de mestrado \citep{logicaResina} foram reconstruídos, com o auxílio dos estudos de Ribeiro \citep{revisaoRibeiro2}. 
	\item[Pseudocontração SRW] Essa operação foi completamente absorvida ao \textit{plug-in} do código feito por Matos, em seu trabalho de conclusão de curso. \citep{logicaMatos}
\end{description} 

Todo o código-fonte está disponível no \href{https://github.com/lsflp/ontology-repair/}{GitHub}, assim como os arquivos
compilados e instruções de compilação e execução.

O programa consiste em uma interface com o usuário pelo terminal. Ele recebe os parâmetros pela linha de comando. A saída é um arquivo OWL com uma ontologia que pode ser aberto no \textit{Protégé}. As informações de entrada e saída podem ser acessadas no \href{https://github.com/lsflp/ontology-repair/blob/master/README.md}{README} do projeto.

\section{Desenvolvimento}

O projeto foi inteiramente feito no IntelliJ versão 2018.2 \footnote{\url{https://www.jetbrains.com/idea/}}. Esse \textit{software} não é gratuito, mas oferece uma versão de uso para estudantes. As dependências foram gerenciadas pelo \textit{Apache Maven} 3.5.2 \footnote{\url{https://maven.apache.org/}}.

Para as lógicas internas, são necessários a \textit{OWL API} 5.1.6 \footnote{\url{http://owlcs.github.io/owlapi/}}, usada para todo o trabalho com os axiomas e o \textit{HermiT Reasoner} 1.3.8 \footnote{\url{http://www.hermit-reasoner.com/}}, um motor de inferências. Para facilitar a entrada dos parâmetros, foi utilizado o \textit{JCommander} 1.58 \footnote{\url{http://jcommander.org}}.

\section{Algoritmo \textit{BlackBox}}

A implementação das quatro operações tem a sua parte principal em um algoritmo \textit{BlackBox}, baseados nos estudos de Resina, que provou sua correção \citep{logicaResina}. Ele é chamado assim pois não precisa de um motor de inferência. É necessário apenas decidir se uma base de conhecimento implica certo axioma, ou se a base é consistente. Ele possui variações para o cálculo do conjunto-resíduo ou para o cálculo do conjunto-\textit{kernel};

Para efeitos de otimização, foi colocado um limite tanto nas filas utilizadas em alguns dos algoritmos abaixo quanto nos conjuntos que serão retornados. Tais limites podem ser definidos pelo usuário, opcionalmente.

\subsection{\textit{BlackBox} para o conjunto-\textit{kernel} para a Contração}

Para o cálculo do conjunto-\textit{kernel}, usado na Contração \textit{Kernel}, foi utilizada uma adaptação do código de Cóbe \citep{revisaoCobe}. Ela consiste em duas funções:

\begin{description}
	\item[\textsc{KernelBlackBox}] É uma função que recebe um conjunto de axiomas $ B $ e uma sentença $ A $. Ela constrói um conjunto minimal de $ B $ que implica $ A $. Essa função é uma adaptação da estudada por Resina \citep{logicaResina}, e se divide em duas partes: expansão, onde são adicionados todos os axiomas de B, caso $ B \vdash A $, a um conjunto $ B' $, definido previamente como vazio; e encolhimento, onde cada elemento $ \beta $ de $ B' $ é analisado individualmente. Caso $ B' \setminus \{\beta\}$ implique $ A $, $ \beta $ é apagado de $ B' $. Logo, o conjunto devolvido é um conjunto minimal de $ B $ que implica $ A $, portanto, um elemento do Kernel. Seu código é o seguinte: \\
	\begin{algorithmic}
		\Function{KernelBlackBox}{B, A}
		\State $ B' \gets \varnothing $
		\If{$ B \vdash A $}
		\State $ B' \gets B $
		\EndIf
		\ForAll{$ \beta \in B' $}
		\If{$ B' \setminus \{\beta\} \vdash A $}
		\State $ B' \gets B' \cup \{\beta\}$
		\EndIf
		\EndFor
		\EndFunction
	\end{algorithmic}

	\item[\textsc{KernelSet}] É uma função que recebe os mesmos parâmetros citados acima, e utiliza a função supracitada. Ela começa com um elemento do Kernel. Assim como o construtor do conjunto-resíduo, ele constrói uma árvore, percorrendo em largura. Para cada elemento $ Hn $ da fila, o algoritmo o remove de $ B $ e, se $ B $ ainda implica $ A $, computa-se o menor subconjunto S de $ B \setminus Hn $ que implica $ A $, e então, tem-se em S um novo elemento do Kernel. Depois disso, $ B $ é restaurado. Seu código está abaixo: \\
	
	\begin{algorithmic}
		\Function{KernelSet}{B, A}
		\If{$ B \nvdash A $}
		\State \Return $ \varnothing $
		\EndIf
		\State $ fila \gets $ fila vazia
		\State $ S \gets $ \Call{KernelBlackBox}{B, A}
		\State $ kernel \gets \{S\} $
		\ForAll{$ s \in S $}
		\State coloque $ s $ em $ fila $
		\EndFor
		\While{$ fila $ não está vazia}
		\State {$ Hn \gets $ o próximo de $ fila $}
		\State $ B \gets B \setminus Hn $
		\If{$ B \vdash A $}
		\State $ S  \gets $ \Call{KernelBlackBox}{B, A}
		\State $ kernel \gets kernel \cup \{S\} $
		\ForAll{$ s \in S $}
		\State coloque $ Hn \cup \{s\} $ em $ fila $
		\EndFor
		\EndIf
		\State $ B \gets B \cup Hn $
		\EndWhile
		\State \Return $ kernel $
		\EndFunction
	\end{algorithmic}
\end{description}

\subsection{\textit{BlackBox} para o conjunto-resíduo}

Para o cálculo do conjunto-resíduo, utilizado na Contração \textit{Partial Meet} e na Pseudocontração SRW, foi utilizada a implementação adaptada por Matos da implementação original de Cóbe \citep{logicaMatos}. Ela consiste em duas funções:

\begin{description}
	\item[\textsc{RemainderBlackBox}] É uma função que recebe um conjunto de axiomas $ B $, uma sentença $ A $ e um conjunto de axiomas $ X $, tal que $ X \nvdash A $, e constrói um elemento do conjunto resíduo $ (B \bot A) $, que contém todos os elementos de $ X $. O conjunto devolvido $ X' $ é tal que $ X' \subseteq X \in B \bot A $. O método começa com $ X $  e acrescenta todos os axiomas de $ B $ que não façam o conjunto resultante implicar $ A $. De acordo como o laço principal é implementado, resultados diferentes podem ser alcançados. No entanto, isso não é um problema, porque a função que chama esta só pede um item do conjunto-resíduo. O seu código segue abaixo: \\
	\begin{algorithmic}
		\Function{RemainderBlackBox}{B, A, X}
		\State $ X' \gets X$
		\ForAll{$ \beta \in B \setminus X $}
		\If{$ X' \cup \{\beta\} \nvdash A $}
		\State $ X' \gets X' \cup \{\beta\} $
		\EndIf 
		\EndFor
		\State \Return $ X' $
		\EndFunction
	\end{algorithmic}
	
	\item[\textsc{RemainderSet}] Usando a função acima, ela constrói o conjunto-resíduo. Seja $ X $ um conjunto, tal que $ X \nvdash A$, inicialmente vazio. Implicitamente, uma árvore é construída. Sua raiz é um elemento do conjunto-resíduo obtido a partir de $ X $, e para cada axioma $ s $ fora desse conjunto, se $ X \cup \{s\} \nvdash A $, o algoritmo cria um nó filho na árvore com um conjunto-resíduo obtido a partir de $ X' = X \cup \{s\} $. Como se deseja apenas gerar o conjunto, a árvore não é construída por completo. Ao invés disso, os elementos são criados como se a árvore fosse percorrida por uma busca em largura, por isso o uso da fila. O seu código segue abaixo: \\
	\begin{algorithmic}
		\Function{RemainderSet}{B, A}
		\State $ fila \gets $ fila vazia
		\State $ S \gets $ \Call{RemainderBlackBox}{B, A, $ \varnothing $}
		\State $ remainder \gets \{S\} $
		\ForAll{$ s \in B \setminus S $}
		\State coloque $ s $ em $ fila $
		\EndFor
		\While{$ fila $ não está vazia}
		\State {$ Hn \gets $ o próximo de $ fila $}
		\If{$ Hn \nvdash A $}
		\State $ S \gets $ \Call{RemainderBlackBox}{B, A, Hn}
		\State $ remainder \gets remainder \cup \{S\} $
		\ForAll{$ s \in B \setminus S $}
		\State coloque $ Hn \cup \{s\} $ em $ fila $
		\EndFor
		\EndIf
		\EndWhile
		\State \Return $ remainder $
		\EndFunction
	\end{algorithmic}
\end{description}

\subsection{\textit{BlackBox} para o conjunto-\textit{kernel} para a Revisão}

O cálculo do conjunto-\textit{kernel} da Revisão \textit{Kernel} é feito a partir de uma construção feita em 2008 \citep{revisaoRibeiro2}. Assim como os outros, consiste em duas funções:

\begin{description}
	\item[\textsc{RevisionKernelBlackBox}] É uma função que recebe o resultado de uma outra operação: a Expansão. Não é importante, no início, se o conjunto de entrada é consistente. Seu funcionamento é semelhante com os \textsc{BlackBox} anteriores. Possui uma parte de expansão e outra de encolhimento. A única diferença aqui, é que em vez de verificar se $ B' \vdash \alpha $ ($ B' $ um conjunto inicialmente vazio), checa-se se $ B' $ é consistente/coerente. A função devolve um conjunto minimal inconsistente. O algoritmo segue abaixo: \\
	\begin{algorithmic}
		\Function{RevisionKernelBlackBox}{$ B + \alpha $}
		\State $ B' \gets \varnothing $
		\ForAll{$ \beta \in B + \alpha $}
		\State $ B' \gets B' \cup \{\beta\} $
		\If{$ B' $ é inconsistente/incoerente}
		\State Pare
		\EndIf
		\EndFor
		\ForAll{$ \epsilon \in B' $}
		\If{$ B' \setminus \{\epsilon\} $ é inconsistente/incoerente}
		\State $ B' \gets B' \setminus \{\epsilon\}  $
		\EndIf
		\EndFor
		\State \Return $ B' $
		\EndFunction
	\end{algorithmic}
	\item[\textsc{RevisionKernelSet}] Essa função também recebe o resultado da Expansão $ B + \alpha $. Diferentemente do conjunto resíduo, aqui a árvore é construída em profundidade. Como todos os algoritmos recursivos, possui um caso base e um geral. O caso base é quando o conjunto de entrada é consistente, e não há nada a ser feito. Para o caso geral, ela chama a função definida acima, e para todos os elementos do conjunto definido abaixo como $ S $, ele remove um dos elementos e tenta achar o conjunto de subconjuntos minimais inconsistentes para essa nova base. Seu código está abaixo: \\
	\begin{algorithmic}
		\Function{RevisionKernelSet}{$ B + \alpha $}
		\If{$ B + \alpha $ é consistente/coerente}
		\State \Return $ \varnothing $
		\EndIf
		\State $ S \gets $ \Call{RevisionKernelBlackBox}{$ B + \alpha $}
		\State $ B' \gets B' \cup \{S\} $
		\ForAll{$ s \in S $}
		\State $ B' \gets B' \cup $ \Call{RevisionKernelSet}{$ B + \alpha \setminus \{\beta\} $}
		\EndFor
		\State \Return $ B' $
		\EndFunction
	\end{algorithmic}
\end{description}

\section{Funções de seleção e incisão}

Duas das operações precisam de uma função de seleção $ \gamma $. A implementação é flexível, a partir de uma interface. A função que está codificada aqui, foi trazida de um trabalho anterior \citep{logicaMatos}, e devolve apenas um elemento do conjunto-resíduo.

As outras duas operações necessitam de uma função de incisão $ \sigma $. A implementação foi feita da mesma maneira, com o uso de uma interface. Ela devolve a união dos elementos do conjunto-\textit{kernel}.

\section{Exemplos da execução}

São feitos aqui alguns exemplos simples para ilustrar a funcionalidade do \textit{plug-in}. Todos os exemplos estão relacionados à ontologia discutida neste trabalho.

\subsection{Contração \textit{Kernel}}

Para a Contração \textit{Kernel}, será usado um exemplo análogo. As classes estudadas aqui são \texttt{Cancao}, \texttt{Pop} e \texttt{SynthPop}. Os axiomas da ontologia são os listados abaixo, e sua representação está na \autoref{img:ck1}.

\begin{itemize}
	\item \texttt{SynthPop} $ \sqsubseteq $ \texttt{Pop}
	\item \texttt{Pop} $ \sqsubseteq $ \texttt{Cancao}
\end{itemize}

Suponha que o subgênero \texttt{SynthPop} evoluiu e está distante do \texttt{Pop}, mas ainda mantém o nome. Portanto, não se deseja mais que ele seja uma subclasse de \texttt{Pop}.

\begin{figure}[H]
	\centering
	\includegraphics[width=0.5\textwidth]{Capitulos/Implementacao/ck1.png}
	\caption{As classes da ontologia de gêneros musicais.}
	\label{img:ck1}
\end{figure}

Com o comando abaixo, a operação é realizada. O resultado está na \autoref{img:ck2}.

\begin{small}
	\texttt{java -jar ontologyrepair-1.0.0-SNAPSHOT.jar -c --core-retainment -i cancao.owl \\ -o output.owl -f "SynthPop SubClassOf Pop"}
\end{small}

\begin{figure}[H]
	\centering
	\includegraphics[width=0.5\textwidth]{Capitulos/Implementacao/ck2.png}
	\caption{O resultado da operação Contração \textit{Kernel}.}
	\label{img:ck2}
\end{figure}


\subsection{Contração \textit{Partial Meet}}

Para a Contração \textit{Partial Meet}, o exemplo utilizado tem as classes \texttt{Musica}, \texttt{Cancao} e \texttt{HinoNacional}, e os seguintes axiomas: 

\begin{itemize}
	\item \texttt{HinoNacional} $ \sqsubseteq $ \texttt{Cancao}
	\item \texttt{Cancao} $ \sqsubseteq $ \texttt{Musica}
\end{itemize}

Sua representação fica como na \autoref{img:cpm1}.

\begin{figure}[H]
	\centering
	\includegraphics[width=0.5\textwidth]{Capitulos/Implementacao/cpm1.png}
	\caption{A estrutura da ontologia de tipos de música.}
	\label{img:cpm1}
\end{figure}

Como definido anteriormente, na verdade, \texttt{Cancao} e \texttt{HinoNacional} são classes na mesma hierarquia, portanto, precisamos remover \texttt{HinoNacional} $ \sqsubseteq $ \texttt{Cancao}. A operação é realizada com o seguinte comando:

\begin{small}
	\texttt{java -jar ontologyrepair-1.0.0-SNAPSHOT.jar -c --relevance -i musica.owl \\ -o output.owl -f "HinoNacional SubClassOf Cancao"}
\end{small}

Após a sua execução, acontece o desejado, mas aparentemente, esse não é o melhor resultado, como pode ser visto na \autoref{img:cpm2}.

\begin{figure}[H]
	\centering
	\includegraphics[width=0.5\textwidth]{Capitulos/Implementacao/cpm2.png}
	\caption{O resultado da operação Contração \textit{Partial Meet}.}
	\label{img:cpm2}
\end{figure}

\subsection{Revisão \textit{Kernel}}

O exemplo usado para a Revisão \textit{Kernel} será simples também. Sejam um fragmento da ontologia de música os axiomas abaixo e representação como na \autoref{img:r1}.

\begin{itemize}
	\item \texttt{FunkAmericano} $ \sqsubseteq $ \texttt{Cancao}
	\item \texttt{FunkBrasileiro} $ \sqsubseteq $ \texttt{FunkAmericano}
\end{itemize}

\begin{figure}[H]
	\centering
	\includegraphics[width=0.5\textwidth]{Capitulos/Implementacao/r1.png}
	\caption{A estrutura da outra ontologia de gêneros musicais.}
	\label{img:r1}
\end{figure}

No entanto, hoje já se classifica o Funk Brasileiro como distinto do Funk Americano. Logo, vamos adicionar à nossa base que os dois subgêneros são disjuntos.

Com o comando abaixo, acontece a operação:

\begin{small}
	\texttt{java -jar ontologyrepair-1.0.0-SNAPSHOT.jar -r -i funk.owl -o output.owl \\ -f "FunkBrasileiro DisjointWith FunkAmericano"}
\end{small}

O resultado é visto na \autoref{img:r2}. Note que a propriedade de disjunção está no produto final.

\begin{figure}[H]
	\centering
	\includegraphics[width=0.45\textwidth]{Capitulos/Implementacao/r2.png}
	\includegraphics[width=0.45\textwidth]{Capitulos/Implementacao/r3.png}
	\caption{O resultado da operação Revisão \textit{Kernel} e a presença da disjunção, na classe \texttt{FunkAmericano}, respectivamente.}
	\label{img:r2}
\end{figure}

\subsection{Pseudocontração SRW}
\label{sect:srw}

Para a Pseudocontração SRW, serão consideradas as classes \texttt{Cantor} e \texttt{Compositor} da ontologia musical, e o seguinte axioma: \texttt{Compositor} $ \sqsubseteq $ \texttt{Cantor}, com a representação exibida na \autoref{img:srw1}.

\begin{figure}[H]
	\centering
	\includegraphics[width=0.5\textwidth]{Capitulos/Implementacao/srw1.png}
	\caption{A estrutura da ontologia sobre ocupações na área de música.}
	\label{img:srw1}
\end{figure}

Ela possui um indivíduo, \texttt{markRonson}, pertencente à classe \texttt{Compositor}, como se mostra na \autoref{img:srw2}.

\begin{figure}[H]
	\centering
	\includegraphics[width=0.5\textwidth]{Capitulos/Implementacao/srw2.png}
	\caption{A instância da ontologia.}
	\label{img:srw2}
\end{figure}

Tudo isso implica que \texttt{markRonson} também é um \texttt{Cantor}, o que está errado. 

A operação pode ser feita com o seguinte comando:

\begin{small}
	\texttt{java -jar ontologyrepair-1.0.0-SNAPSHOT.jar -srw -i pessoa.owl -o output.owl \\ -f "markRonson Type: Cantor"}
\end{small}

Depois da operação, temos que \texttt{Compositor} deixa de ser subclasse de \texttt{Cantor}, e então, \texttt{markRonson} não pertence mais à classe \texttt{Cantor}, como se observa na \autoref{img:srw3}.

\begin{figure}[H]
	\centering
	\includegraphics[width=0.5\textwidth]{Capitulos/Implementacao/srw3.png}
	\caption{O resultado da operação Pseudocontração SRW.}
	\label{img:srw3}
\end{figure}

	
	\chapter{Análise de desempenhos}
	\chapter*{Agradecimentos}
	
	% Bibliografia
	\backmatter \singlespacing   % espaçamento simples
	\bibliographystyle{plainnat-ime} % citação bibliográfica textual
	\bibliography{monografia.bib}  % associado ao arquivo: 'bibliografia.bib'

	 
\end{document}