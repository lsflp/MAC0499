\documentclass[12pt,letterpaper]{article}
\usepackage{amsmath,amsthm,amsfonts,amssymb,amscd}
\usepackage[table]{xcolor}
\usepackage[margin=2cm]{geometry}
\usepackage{graphicx}
\usepackage{multicol}
\usepackage{mathtools}
\usepackage[default]{lato}
\usepackage[T1]{fontenc}
\usepackage[utf8]{inputenc}
\usepackage[brazil]{babel}
\usepackage{cite}
\usepackage[pdftex]{hyperref}
\renewcommand*{\ttdefault}{txtt}
\newlength{\tabcont}
\setlength{\parindent}{0.0in}
\setlength{\parskip}{0.05in}

\begin{document}
	
	\large
	
	\title{Projeto de Revisão de Crenças em Lógica de Descrição}
	\author{\textbf{Aluno:} Luís Felipe de Melo Costa Silva (\texttt{luis.melo.silva@usp.br})\\ 
		\textbf{Supervisora:} Renata Wassermann (\texttt{renata@ime.usp.br})\\ \\
		Instituto de Matemática e Estatística, Universidade de São Paulo}
	
	\maketitle
	
	\section{Resumo}
	
	Este trabalho busca estudos que viabilizem o reparo de uma ontologia inconsistente no Protégé. Para tal, será feita uma pesquisa, e com os conhecimentos adquiridos, atualizar os \textit{plug-ins} que cuidam do reparo e contemplar novas operações serão os objetivos.
	
	Serão selecionadas referências textuais na área de Lógica, em especial sobre Lógicas de Descrição, Ontologias e Revisão de Crenças, além de materiais que permitam a construção do \textit{plug-in} propriamente dito. Tal material vai embasar a teoria do projeto e auxiliar na resolução do problema.
	
	Todo esse esforço é realizado para verificar a hipótese de que, aplicando técnicas de Revisão de Crenças, seja possível reparar uma ontologia inconsistente através de um algoritmo. Queremos também verificar o desempenho de cada \textit{plug-in}, tanto algorítmico quanto dos resultados. 
	
	
	\textbf{Palavras-chave:} Lógica, Lógicas de Descrição, Ontologias, Reparo, Revisão de Crenças, Protégé.
	
	
	\clearpage
	
	\section{Justificativa} 
	
	Ontologias são uma especificação explícita de uma conceitualização \cite{gruber1995toward}. Isso quer dizer que as ontologias são usadas para explicitar os conceitos e as relações possíveis dentro de algum domínio. Um domínio pode ser qualquer área do conhecimento, como por exemplo, a área médica, a enologia, entre outros. Para montar as ontologias, podem ser usadas Lógicas de Descrição. Elas são sub-linguagens da Lógica de Primeira Ordem e possuem uma expressividade ótima para modelagem. A caracterização dos conceitos e relações de uma ontologia pode ser feita a partir de axiomas em Lógicas de Descrição. \\
	
	Os domínios do conhecimento se expandem com o passar do tempo. Para que uma ontologia permaneça útil, é necessário atualizá-la. No entanto, existe um obstáculo a ser enfrentado. Pode acontecer que a inclusão de um novo axioma torne a ontologia inconsistente, ou seja, tal axioma pode entrar diretamente em conflito com um outro que esteja presente no sistema ou com alguma inferência que possa ser feita. Para que a consistência da ontologia seja restaurada, e assim ela continue sendo utilizada, podem ser usadas técnicas de Revisão de Crenças\cite{gardenfors2003belief} \cite{gardenfors1992belief}. \\
	
	Uma interface utilizada para a construção de ontologias é o \href{https://protege.stanford.edu/}{Protégé}. Com ele, é possível analisar praticamente qualquer domínio. Todos os dados ficam em algum tipo de arquivo, sendo a OWL (\textit{Ontology Web Language}) uma das extensões utilizadas. Utilizando o Protégé e uma ontologia construída, ele é capaz até de fazer buscas, como num banco de dados. Com o auxílio de \textit{plug-ins} \cite{resina2014belief} \cite{ribeiro2008}, o poder do programa aumenta. Existem \textit{plug-ins} para inferências, e implementações para Revisão de Crenças (feitas em 2008, 2010 e 2014), que estão desatualizadas. \\ 
	
	O objetivo deste trabalho é descobrir como reparar uma ontologia, no Protégé, que se torna inconsistente após a entrada de um novo fato. Deseja-se comprovar se, utilizando técnicas de Revisão de Crenças, tais como a Expansão, a Contração e a Revisão, uma ontologia pode ser reparada. Ou seja, fazendo um estudo sobre as técnicas de Revisão de Crenças, é possível implementar um plug-in para o Protégé que faça o reparo nas ontologias? Além disso, quer-se atualizar implementações prévias para as novas versões do Protégé e implantar algumas operações que ainda não foram contempladas.  \\
	
	\clearpage
	
	\section{Metodologia}
	
	\begin{enumerate}
		\item Seleção bibliográfica e estudo do material a ser aplicado: Textos de sobre Lógicas de Descrição, Ontologias e Revisão de Crenças. 
		\item Estudo de plug-ins para o Protégé.
		\item Descrição e Aplicação desse material: Os textos sobre Lógicas de Descrição e Ontologias serão utilizados para familiarizar o leitor com alguns termos da área. Os de Revisão de Crenças servirão como apoio principal ao desenvolvimento do trabalho. Já os estudos de plug-ins para o Protégé servirão para descobrir quais as técnicas e estruturas de dados serão usadas afim de criar um novo plug-in. 
		\item Implementação: A partir dos estudos teóricos e com os plug-ins, tentaremos construir um para o Protégé que repara ontologias.
		\item Análise e Discussão dos resultados: Com os estudos feitos, vamos verificar se o reparo das ontologias é satisfatório, ou seja, uma análise dos limites da aplicação e uma de desempenho da implementação.
	\end{enumerate}
	
	\section{Cronograma}

	Seguindo a numeração acima, teremos:
	
	\begin{center}
		\begin{tabular}{|c|c|}
			\cline{1-2}
			\textbf{Etapa} & \textbf{Período de execução previsto}  \\ \cline{1-2}
			1 & Maio \\ \cline{1-2}
			2 & Maio \\ \cline{1-2}
			3 & Junho e julho \\ \cline{1-2}
			4 & Agosto e setembro \\ \cline{1-2}
			5 & Outubro e novembro \\ \cline{1-2}
			Escrita da Monografia & Maio até novembro  \\ \cline{1-2}
		\end{tabular}
	\end{center}
	
	\bibliographystyle{plain}
	\bibliography{proposta}
	 
\end{document}